\documentclass[10pt,fleqn]{article} % Default font size and left-justified equations
\usepackage[%
    pdftitle={Introduction à l'algorithmique et à la programmation},
    pdfauthor={Xavier Pessoles}]{hyperref}

\newcommand{\repRel}{../../../..}
\input{\repRel/Style/packages}
\input{\repRel/Style/new_style}
\input{\repRel/Style/macros_SII}
\input{\repRel/Style/environment}


\newcommand{\macrocomp}{macro_competences}
\newcommand{\comp}{competences}
\newcommand{\td}{fichier_td}
\newcommand{\repExo}{dossier}
\newcommand{\repStyle}{\repRel/Style}

\def\xxYCartouche{-2.25cm}
\def\xxYongletGarde{.5cm}
\def\xxYOnget{.9cm}

\begin{document}

\livrettrue
\graphicspath{{\repStyle/png/}}
\input{\repStyle/Info_01_Entete}
\graphicspath{{\repStyle/png/}{\repRel/Informatique/Semestre_01/01_Bases/Cours/images/}}
%\input{\repRel/Informatique/Semestre_01/01_Bases/Cours/01_Bases.tex}


\def\xxcompetences{}
\def\xxfigures{}
\def\xxchapitre{}
\def\xxtitreexo{Quelques algorithmes de tris}
\def\xxsourceexo{}
\def\xxactivite{TP 10 \ifprof  -- Corrigé \else \fi}
\input{\repRel/Style/pagegarde_TD}


\proftrue
\vspace{5cm}
\pagestyle{fancy}
\thispagestyle{plain}

\graphicspath{{\repStyle/png/}{\repRel/Informatique/Exercices/S1_07_Images/01_TraitementPGM/}}
%\input{\repRel/Informatique/Exercices/S1_07_Images/01_TraitementPGM.tex}


%
%\activite{Recherche d'un plus court chemin par un algorithme glouton}
%Commencez par télécharger la trame du fichier \texttt{.py} sur le site de la classe : 
%\url{https://mpsilamartin.github.io/info/TP/08_RechercheChemin.py}.


\activite{Tri par comptage}
\renewcommand{\td}{04_TriComptage}
\renewcommand{\repExo}{\repRel/Informatique/Exercices/S1_08_Tris}
\graphicspath{{\repStyle/png/}{\repExo/\td}}
\input{\repExo/\td}

\activite{Tri fusion}
\renewcommand{\td}{02_TriFusion}
\renewcommand{\repExo}{\repRel/Informatique/Exercices/S1_08_Tris}
\graphicspath{{\repStyle/png/}{\repExo/\td}}
\input{\repExo/\td/\td}

\activite{Comparaison des tris}

\begin{obj}
L'objectif est de comparer les différents tris en comptant le nombre d'opérations élémentaires réalisées.
\end{obj}

\question{Copier -- coller les algorithmes de tri par sélection et de tri rapide réaliser avec Capytale.}

\begin{lstlisting}
def compteur():
    global C
    C = C+1	
\end{lstlisting}

\question{Copier -- coller la fonction ci-dessus dans votre script et ajouter l'appel \texttt{compteur()} une fois pour chacun des tris, au niveau de l'instruction étant réalisée le plus grand nombre de fois.}


Soit les instructions suivantes.
\begin{lstlisting}
import matplotlib.pyplot as plt
import random as rd
    
les_i = []
les_selection = []

for i in range(1,100,10):
    les_i.append(i)
    L = [rd.randrange(0,i) for x in range(i)]

    C = 0
    tri_comptage(L.copy())
    les_comptage.append(C)

plt.plot(les_i,les_comptage,label='Comptage')
plt.grid()
plt.legend()
plt.show()
\end{lstlisting}

\question{Vérifier que ces lignes sont fonctionnelles. Commenter ces lignes par blocs d'instructions.}


\question{Adapter ces lignes pour comparer chacune des méthodes de tri. Conclure}





\end{document}




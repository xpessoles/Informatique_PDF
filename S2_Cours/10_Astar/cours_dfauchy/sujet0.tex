% Options for packages loaded elsewhere
\PassOptionsToPackage{unicode}{hyperref}
\PassOptionsToPackage{hyphens}{url}
%
\documentclass[
]{article}
\usepackage{lmodern}
\usepackage{amssymb,amsmath}
\usepackage{ifxetex,ifluatex}
\ifnum 0\ifxetex 1\fi\ifluatex 1\fi=0 % if pdftex
  \usepackage[T1]{fontenc}
  \usepackage[utf8]{inputenc}
  \usepackage{textcomp} % provide euro and other symbols
\else % if luatex or xetex
  \usepackage{unicode-math}
  \defaultfontfeatures{Scale=MatchLowercase}
  \defaultfontfeatures[\rmfamily]{Ligatures=TeX,Scale=1}
\fi
% Use upquote if available, for straight quotes in verbatim environments
\IfFileExists{upquote.sty}{\usepackage{upquote}}{}
\IfFileExists{microtype.sty}{% use microtype if available
  \usepackage[]{microtype}
  \UseMicrotypeSet[protrusion]{basicmath} % disable protrusion for tt fonts
}{}
\makeatletter
\@ifundefined{KOMAClassName}{% if non-KOMA class
  \IfFileExists{parskip.sty}{%
    \usepackage{parskip}
  }{% else
    \setlength{\parindent}{0pt}
    \setlength{\parskip}{6pt plus 2pt minus 1pt}}
}{% if KOMA class
  \KOMAoptions{parskip=half}}
\makeatother
\usepackage{xcolor}
\IfFileExists{xurl.sty}{\usepackage{xurl}}{} % add URL line breaks if available
\IfFileExists{bookmark.sty}{\usepackage{bookmark}}{\usepackage{hyperref}}
\hypersetup{
  hidelinks,
  pdfcreator={LaTeX via pandoc}}
\urlstyle{same} % disable monospaced font for URLs
\usepackage{longtable,booktabs}
% Correct order of tables after \paragraph or \subparagraph
\usepackage{etoolbox}
\makeatletter
\patchcmd\longtable{\par}{\if@noskipsec\mbox{}\fi\par}{}{}
\makeatother
% Allow footnotes in longtable head/foot
\IfFileExists{footnotehyper.sty}{\usepackage{footnotehyper}}{\usepackage{footnote}}
\makesavenoteenv{longtable}
\usepackage{graphicx}
\makeatletter
\def\maxwidth{\ifdim\Gin@nat@width>\linewidth\linewidth\else\Gin@nat@width\fi}
\def\maxheight{\ifdim\Gin@nat@height>\textheight\textheight\else\Gin@nat@height\fi}
\makeatother
% Scale images if necessary, so that they will not overflow the page
% margins by default, and it is still possible to overwrite the defaults
% using explicit options in \includegraphics[width, height, ...]{}
\setkeys{Gin}{width=\maxwidth,height=\maxheight,keepaspectratio}
% Set default figure placement to htbp
\makeatletter
\def\fps@figure{htbp}
\makeatother
\setlength{\emergencystretch}{3em} % prevent overfull lines
\providecommand{\tightlist}{%
  \setlength{\itemsep}{0pt}\setlength{\parskip}{0pt}}
\setcounter{secnumdepth}{-\maxdimen} % remove section numbering
\ifluatex
  \usepackage{selnolig}  % disable illegal ligatures
\fi

\author{}
\date{}

\begin{document}

\emph{\textbf{TD 11-3}}

\emph{\textbf{A star}}

\begin{enumerate}
\def\labelenumi{\arabic{enumi}.}
\item
  A star
\end{enumerate}

Contexte

On souhaite trouver le plus court chemin à réaliser sur une image en
noir et blanc entre un départ et une arrivée. Les pixels noirs seront
des obstacles, les blancs des cases accessibles. Voici des exemples de
chemins que vous pourrez trouver dans la dernière partie de ce TD~:

\includegraphics[width=4.1039in,height=1.48575in]{media/image1.png}

\includegraphics[width=2.73377in,height=2.63687in]{media/image2.png}

Nous allons donc~:

\begin{itemize}
\item
  Importer des fonctions d'affichage prédéfinies
\item
  Réaliser la grille/image de départ
\item
  Créer un dictionnaire représentant le graphe du domaine d'étude
\item
  Trouver le plus court chemin à l'aide de l'algorithme de Dijkstra
\item
  Faire de même avec l'algorithme A-star
\item
  Comparer performances de ces algorithmes
\end{itemize}

Pour rappel~:

\begin{itemize}
\item
  L'algorithme de Dijkstra permet de trouver le plus court chemin dans
  un graphe en «~avançant~» par «~arcs de cercles~» depuis le départ,
  jusqu'à atteindre l'arrivée.
\item
  L'algorithme A-star privilégie dans ses choix, les cases dont
  l'heuristique (distance depuis le début + distance à vol d'oiseau
  jusqu'à l'arrivée) est la plus faible. Dans ce TD, il cherchera donc
  toujours à se diriger vers l'arrivée.
\end{itemize}

Prise en main du code élèves

Afin d'assurer un fonctionnement rapide sur tous les ordinateurs, je
vous mets à disposition un dossier à télécharger COMPLETEMENT, soit le
dossier contenant tous les fichiers et images, et non les fichiers pris
séparément.

\includegraphics[width=1.62987in,height=0.84816in]{media/image3.png}

Sans ouvrir le dossier, faite juste «~Télécharger -- Téléchargement
direct~» puis mettez ce dossier dans votre répertoire personnel.

\href{https://www.dropbox.com/sh/pe158jwai7yqkol/AAA3N3sptIYvkH64A90r2DUqa?dl=0}{\textbf{LIEN}}

Si le téléchargement est sous forme de Rar, Zip\ldots{} Pensez à
dézipper l'archive afin d'avoir le dossier voulu~!

Vous avez à disposition 5 fichiers Python~:

\begin{longtable}[]{@{}ll@{}}
\toprule
\begin{minipage}[b]{0.47\columnwidth}\raggedright
11-3 - 1 - Affichage\strut
\end{minipage} & \begin{minipage}[b]{0.47\columnwidth}\raggedright
Import des librairies numpy et matplotlib

Création de 3 fonctions d'affichage~:

Affiche(fig,im,grille)~: Affiche l'image d'étude

Affiche\_Save(fig,im,grille,chemin)~: Enregistre l'image pour créer des
animations (sans affichage sinon bug de redimensionnement)

Affiche\_Degrade(Fig,Tab)~: Affiche avec un dégradé les distances de
chaque pixel depuis le départ sur une nouvelle image\strut
\end{minipage}\tabularnewline
\midrule
\endhead
11-3 - 2 - Grille - Elèves & Code prérempli à compléter pour créer la
grille de départ\tabularnewline
11-3 - 3 - Dico - Elèves & A compléter\tabularnewline
11-3 - 4 - Dijkstra -- Elèves & A compléter\tabularnewline
11-3 - 4 - A star - Elèves & A compléter\tabularnewline
\bottomrule
\end{longtable}

\begin{enumerate}
\def\labelenumi{\arabic{enumi}.}
\item
  Télécharger et exécuter les fichiers Affichage et Grille dans l'ordre
\end{enumerate}

A ce stade, vous devriez voir apparaître le domaine d'étude sous la
forme suivante~:

\includegraphics[width=2.61039in,height=2.61039in]{media/image4.png}

La grille est remplie de cases blanches (accessibles), on voit
apparaître le point de départ en rouge, le point d'arrivée en bleu et
une case obstacle en noir.

Création de la grille de départ

Vous travaillerez dans le fichier nommé «~11-3 - 2 - Grille -
Elèves.py~».

On souhaite réaliser la grille de départ suivante~:

\includegraphics[width=6.3in,height=6.26875in]{media/image5.png}

\begin{enumerate}
\def\labelenumi{\arabic{enumi}.}
\setcounter{enumi}{1}
\item
  Compléter/Modifier le code afin de réaliser cette image de départ
\end{enumerate}

Réalisation du dictionnaire des voisins

~

Les mouvements possibles pour se déplacer d'une case à l'autre sont les
8 directions~:

\begin{itemize}
\item
  ↑→↓← de distance \(1\)
\item
  ↖↗↘↙ de distance \(\sqrt{2}\)
\end{itemize}

On souhaite réaliser un dictionnaire nommé «~Dico\_Voisins~»
représentant le graphe du système tel que~:

\begin{itemize}
\item
  Les clés sont des tuples de type (ligne,colonne) de chaque pixel pour
  chaque case/pixel accessible (pas noire). Exemple d'une portion des
  clés du dictionnaire~:
\end{itemize}

\includegraphics[width=5.78261in,height=0.88409in]{media/image6.png}

\begin{itemize}
\item
  Les valeurs associées à chaque clé/case sont des listes contenant des
  sous listes (une par case voisine accessible) de deux éléments du type
  {[}clé case,distance{]}. Exemple~:
\end{itemize}

\includegraphics[width=4.94655in,height=0.43511in]{media/image7.png}

Vous travaillerez dans le fichier nommé «~11-3 - 3 - Dico - Elèves.py~».

\begin{enumerate}
\def\labelenumi{\arabic{enumi}.}
\item
  Créer une fonction Test\_Pix(P1,P2) qui renvoie le booléen répondant à
  la question «~Le pixel P1 est égal au pixel P2~»
\end{enumerate}

Vérifier~:

\includegraphics[width=3.30435in,height=0.91278in]{media/image8.png}

\begin{enumerate}
\def\labelenumi{\arabic{enumi}.}
\setcounter{enumi}{1}
\item
  Mettre en place le code permettant de créer le dictionnaire
  Dico\_Voisins contenant pour valeurs de chaque case accessible des
  listes vides
\end{enumerate}

Remarque~: Penser qu'une case accessible est «~non noire~», et
non\ldots{} blanche.

Vous devriez voir~:

\includegraphics[width=2.64454in,height=1.16522in]{media/image9.png}

\begin{enumerate}
\def\labelenumi{\arabic{enumi}.}
\setcounter{enumi}{2}
\item
  Mettre en place la fonction Couples\_Voisins(l,c) qui, pour la case à
  la ligne l et la colonne c, ajoute dans sa liste dans Dico\_Voisins,
  les couples {[}Case\_i,Di{]} des cases voisins accessibles (max 8
  cases possibles)
\end{enumerate}

Remarque~:

\begin{itemize}
\item
  On évitera le test «~in Dico\_Voisins~» pour vérifier qu'une case est
  accessible, qui coûte bien plus cher que de vérifier la couleur non
  noire d'un pixel
\item
  Penser que sur tous les bords de l'image, il n'y a plus 8
  voisins\ldots{} Cela se traduira par des conditions sur li et ci des
  cases voisines «~Case\_i~»
\end{itemize}

Vérifier~:

\includegraphics[width=5.37203in,height=0.8in]{media/image10.png}

\begin{enumerate}
\def\labelenumi{\arabic{enumi}.}
\item
  Mettre en place les lignes de code permettant de remplir Dico\_Voisins
\end{enumerate}

Vérifier~:

\includegraphics[width=6.3in,height=1.73056in]{media/image11.png}

Mise en place de l'algorithme de Dijkstra

Vous travaillerez dans le fichier nommé «~11-3 - 4 - Dijkstra -
Elèves.py~».

Ce fichier contient un paragraphe d'initialisation~:

\begin{longtable}[]{@{}ll@{}}
\toprule
\begin{minipage}[b]{0.47\columnwidth}\raggedright
Distances\strut
\end{minipage} & \begin{minipage}[b]{0.47\columnwidth}\raggedright
Array aux mêmes dimensions que l'image traitée, tel que
Distances{[}l,c{]} est la distance du chemin menant à la case concernée
depuis le départ

Il est initialisé avec des valeurs infinies\strut
\end{minipage}\tabularnewline
\midrule
\endhead
Reste & Copie du dictionnaire Dico\_Voisins. On enlèvera les cases
traitées de ce dictionnaire et l'algorithme se terminera au plus tard,
quand ce dictionnaire est vide\tabularnewline
\begin{minipage}[t]{0.47\columnwidth}\raggedright
Provenances\strut
\end{minipage} & \begin{minipage}[t]{0.47\columnwidth}\raggedright
Dictionnaire des provenances de chaque case. Initialisé vide, il
contiendra à la fin de la réalisation de l'algorithme des clés et
valeurs qui seront respectivement le nom de la case concernée et la case
qui a permis d'y accéder lors de l'exécution de l'algorithme.

Par exemple, si Provenances{[}(1,1){]}=(0,0), c'est que pour atteindre
la case (1,1) avec un chemin depuis le départ de longueur
Distances{[}1,1{]}, la case précédente de ce chemin est (0,0)\strut
\end{minipage}\tabularnewline
\bottomrule
\end{longtable}

Pour rappel, départ et arrivée ont été définies précédemment
(ld,cd,la,ca).

Nous allons maintenant programmer l'algorithme de Dijkstra
(rappelez-vous le TD 11-1~), dont nous allons rappeler les grandes
étapes.

Tant que \textbf{S} n'est pas l'arrivée, qu'il reste des stations dans
\textbf{Reste} et que Distance{[}lS,cS{]} est différente de l'infini
(cette condition permet de gagner du temps si \textbf{Arrivee} n'est pas
accessible depuis \textbf{Depart})

\begin{itemize}
\item
  \textbf{S} ← Station parmi \textbf{Reste} ayant la distance minimum
  dans \textbf{Distances}
\item
  Mise à jour de \textbf{Reste} (retirer S)
\item
  \textbf{Voisins} ← Dictionnaire des stations de \textbf{Reste}
  voisines de \textbf{S}
\item
  Traitement des voisins~: Pour chaque voisin V, si la distance
  Depart→S→V \textless{} Depart→V actuellement stockée dans
  \textbf{Distances}~:

  \begin{itemize}
  \item
    Mise à jour de \textbf{Distances} avec cette nouvelle distance
  \item
    Mise à jour de \textbf{Provenances}~afin d'indiquer que S est le
    prédécesseur de V
  \end{itemize}

  \begin{enumerate}
  \def\labelenumi{\arabic{enumi}.}
  \item
    Mettre en place cet algorithme
  \end{enumerate}
\end{itemize}

On remarquera que si la case arrivée n'est pas accessible, Reste sera
vide, mais en plus, Distances{[}la,ca{]} sera toujours infinie.

\begin{enumerate}
\def\labelenumi{\arabic{enumi}.}
\setcounter{enumi}{1}
\item
  Mettre en place le code permettant soit d'afficher que le chemin
  n'existe pas, soit de remonter le chemin menant à l'arrivée, en
  affichant dans la console les cases empruntées, la distance parcourue,
  et le nombre d'itérations réalisées.
\end{enumerate}

Vérifier~:

\includegraphics[width=6.3in,height=1.70208in]{media/image12.png}

\begin{enumerate}
\def\labelenumi{\arabic{enumi}.}
\setcounter{enumi}{2}
\item
  Mettre en place le code permettant d'afficher le chemin trouvé en vert
  sur l'image
\end{enumerate}

\includegraphics[width=2.496in,height=2.43467in]{media/image13.png}

\begin{enumerate}
\def\labelenumi{\arabic{enumi}.}
\setcounter{enumi}{3}
\item
  En utilisant la fonction Affiche\_Degrade, afficher les distances des
  pixels depuis le départ et les pixels traités par l'algorithme
\end{enumerate}

\includegraphics[width=3.22535in,height=2.52765in]{media/image14.png}

Mise en place de l'algorithme de A-star

Vous travaillerez dans le fichier nommé «~11-3 - 5 - A star -
Elèves.py~».

Dijkstra choisissait \(s\) comme la station la plus proche du départ. On
propose de prendre en compte l'heuristique «~Distances à vol d'oiseau~»
entre \(s\) et l'arrivée.

\[f\left( s \right) = d\left( s \right) + d^{'}\left( s,s_{\text{fin}} \right)\]

\begin{enumerate}
\def\labelenumi{\arabic{enumi}.}
\setcounter{enumi}{4}
\item
  Créer la fonction f(Case) renvoyant le calcul attendu
\item
  Copier/coller le code Dijkstra et modifier ce qu'il faut afin de
  réaliser l'algorithme A star
\item
  Comparer les algorithmes Dijkstra et A star
\end{enumerate}

Pour aller plus loin

Vous avez à disposition trois images :

\begin{longtable}[]{@{}lll@{}}
\toprule
Labyrinthe petit & Labyrinthe grand & Réseau\tabularnewline
\midrule
\endhead
\includegraphics[width=1.07976in,height=1.34758in]{media/image15.png} &
\includegraphics[width=1.90184in,height=0.63598in]{media/image16.png} &
\includegraphics[width=1.65571in,height=1.57097in]{media/image17.png}\tabularnewline
\bottomrule
\end{longtable}

Vous trouverez dans le dossier télécharger en début de TD les 3 images
ci-dessus, partagées sous les formats BMP (ouverture avec matplotlib) et
array (ouverture avec numpy en cas de bug matplotlib).

Pour chacun de ces 3 cas~:

\begin{enumerate}
\def\labelenumi{\arabic{enumi}.}
\setcounter{enumi}{7}
\item
  Créer un nouveau fichier «~11-2 -- 2 -- Nom\_à\_choisir.py~»
\item
  ~Sous Python, ouvrir l'image et l'afficher
\item
  Créer une fonction f\_NB(im) qui renvoie une nouvelle image en noir et
  blanc ({[}0,0,0{]} ou {[}255,255,255{]}) à partir de l'image im
\end{enumerate}

Attention~: les labyrinthes seront aussi transformés en nior et blanc
(je n'ai pas vérifié chaque pixel pour être certain que les triplets
sont bien composés uniquement de 0 ou de 255.

\begin{enumerate}
\def\labelenumi{\arabic{enumi}.}
\setcounter{enumi}{10}
\item
  Transformer l'image en noir et blanc et l'afficher
\item
  Choisir et ajouter les points de départ et d'arrivée sur l'image
\item
  Faire tourner les algorithmes Dijkstra et A-star
\item
  Apprécier les résultats et comparer l'efficacité des algorithmes
\end{enumerate}

\end{document}

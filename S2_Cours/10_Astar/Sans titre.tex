Torseur du contact ponctuel en A de $S_0\to S_2$;Pas d'erreur de syntaxe;Formule de changement de point;Torseur en H
Torseur du contact ponctuel en B de $S_0\to S_2$;Pas d'erreur de syntaxe;Formule de changement de point;Torseur en H
Torseur du câble en H de $S_1\to S_2$;Pas d'erreur de syntaxe
Pas d'erreur de syntaxe
Q1
Moment de l'action en A exprimé en H
Moment de l'action en B exprimé en H
Moment de l'action en E exprimé en H
Moment du poids exprimé en H
Equation véctorielle du TMS en H
Equation scalaire obtenue
Q2
Condition
Inégalité
Q3
Conclusion 1
Conclusion 2
Q4
Conclusion vis-à-vis du cahier des charge

Q5
Construction graphique
Estimation de ΔxA
Q6
AN sur R(R2->4)

Q7
Solide soumis à deux glisseurs
Action driectement opposées
Q8
Solide soumis à trois glisseurs
Point de concours I4
Triangle des forces
Q9
Solide soumis à trois glisseurs
Point de concours I1
Triangle des forces
Q10
Résolution graphique
AN
Q11
Problème 3D
BAME : 5 AM
Composantes tangentielles et normales
Action de la pesanteur
Q12
TRS suivant z
Loi de Coulomb
Inégalité
Q13
Application numérique
Conclusion
Q14
Vérification 1
Vérification 2
Vérifixation 3
Résolution graphique
Application numérique
Q15
Expression de OP
Expression de r
Q16
Torseur d'AM local assimilé à un glisseur

Q17
Expression du torseur d'action méca locale dT3->2
Expression de dN
Q18
Expression de l'action tangentielle dT3->2
Expression du torseur d'action méca locale dT3->2 en 0
Q19
Expression sous forme intégralle
Calcul d'intégrale
Expression de R2->3
Calcul de M/0(2->3)
Q20
Conclusion

% Déclaration des titres
% -------------------------------------


\graphicspath{{../../../style/png/}{images/}{../../../Informatique/exercices/11_plot/generalites/}}
\lstinputpath{{../../../Informatique/exercices/11_plot/generalites/}}

\def\discipline{Informatique}
\def\xxtete{Informatique}

\def\classe{\textsf{MPSI}}
\def\xxnumpartie{Semestre 1}
\def\xxpartie{}
\def\xxdate{12 Mai 2022}

\def\xxchapitre{1 et 2}
\def\xxnumchapitre{1 et 2}
\def\xxnomchapitre{Représentation des nombres}
\def\xxnumactivite{17}

\def\xxposongletx{2}
\def\xxposonglettext{1.45}
\def\xxposonglety{19}%16

\def\xxonglet{\textsf{Cycle 01}}
\def\xxauteur{\textsl{E. Durif -- X. Pessoles \\ J.P. Berne }}


\def\xxpied{%
Cycle \xxnumpartie -- \xxpartie\\
Chapitre \xxnumchapitre -- \xxactivite -\xxnumactivite -- \xxnomchapitre%
}

\setcounter{secnumdepth}{5}
\chapterimage{}
\def\xxfigures{}

\def\xxcompetences{%
\textsl{%
%\vspace{-.5cm}
\textbf{Savoirs et compétences :}\\
\vspace{-.1cm}
\begin{itemize}[label=\ding{112},font=\color{ocre}]
\item BDD.C1 : Utiliser une application offrant une interface graphique pour créer une base de données et l'alimenter
\item BDD.C2 : Utiliser une application offrant une interface graphique pour lancer des requêtes sur une base de données
\item BDD.C5 : Concevoir une base constituée de plusieurs tables, et utiliser les jointures symétriques pour effectuer des requêtes croisées
\item BDD.S3 : Opérateurs spécifiques de l'algèbre relationnelle : projection, sélection (ou restriction), renommage, jointure, produit et division cartésiennes ; fonctions d'agrégation : min, max, somme, moyenne, comptage.
\end{itemize}
}}


%Infos sur les supports
\def\xxtitreexo{Représentation des nombres}
\def\xxsourceexo{\hspace{.2cm} \footnotesize{\textbf{Sources : }
}}
%\def\xxtitreexo{Titre EXO}
%\def\xxsourceexo{\hspace{.2cm} \footnotesize{Source EXO}}


%---------------------------------------------------------------------------



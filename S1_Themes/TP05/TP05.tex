%%%% Paramétrage du cours %%%%
\def\xxactivite{\ifprof TP -- Corrigé  \else  TP \fi}
\def\xxauteur{\textsl{É. Durif -- X. Pessoles -- J.-P. Berne}}
%\fichefalse
%\proftrue
%\tdfalse
%\courstrue



\def\xxYCartouche{-2.25cm}
\def\xxYongletGarde{.5cm}
\def\xxYOnget{.9cm}


\def\xxfigures{
	%\includegraphics[width=.6\linewidth]{fig_00}
}%figues de la page de garde


\input{../../../style/pagegarde_info_tp}

\setlength{\columnseprule}{.1pt}

\pagestyle{fancy}
\thispagestyle{plain}


\vspace{3.5cm}

\def\columnseprulecolor{\color{ocre}}
\setlength{\columnseprule}{0.4pt} 

%%%%%%%%%%%%%%%%%%%%%%%

\setcounter{exo}{0}
\vspace{3cm}



\vskip1cm \noindent\textbf{Consignes}

\input{../../../Informatique/Exercices/consignes/consignes_tp5}

\vskip1cm

\eject \activite{Recherche par dichotomie d'un élément dans une liste triée}

\input{../../../Informatique/Exercices/S1_04_AlgorithmesDichotomiques/Dichotomie}


 \activite{Recherche d'un zéro d'une fonction}

\input{../../../Informatique/Exercices/S1_04_AlgorithmesDichotomiques/Zeros_fonctions}


\activite{Valeur d'un polynôme en un point par plusieurs méthodes}

\input{../../../Informatique/Exercices/S1_04_AlgorithmesDichotomiques/Horner}
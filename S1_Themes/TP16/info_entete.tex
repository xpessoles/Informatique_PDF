% Déclaration des titres
% -------------------------------------


\graphicspath{{../../../style/png/}{images/}{../../../Informatique/exercices//}}
\lstinputpath{{../../../Informatique/exercices//}}

\def\discipline{Informatique}
\def\xxtete{Informatique}

\def\classe{\textsf{MPSI}}
\def\xxnumpartie{Semestre 1}
\def\xxpartie{}
\def\xxdate{14 Avril 2022}

\def\xxchapitre{4}
\def\xxnumchapitre{4}
\def\xxnomchapitre{Complexité}
\def\xxnumactivite{16}

\def\xxposongletx{2}
\def\xxposonglettext{1.45}
\def\xxposonglety{19}%16

\def\xxonglet{\textsf{Cycle 01}}
\def\xxauteur{\textsl{E. Durif -- X. Pessoles \\ J.-P. Berne }}


\def\xxpied{%
\xxnumpartie -- \xxnomchapitre \\
%Cycle \xxnumpartie -- \xxpartie\\
%Chapitre \xxnumchapitre -- \xxactivite -\xxnumactivite -- \xxnomchapitre%
%Chapitre \xxnumchapitre -- 
\xxactivite \xxnumactivite %
}

\setcounter{secnumdepth}{5}
\chapterimage{Fond_DB}
\def\xxfigures{}

\def\xxcompetences{%
\textsl{%
%\vspace{-.5cm}
\textbf{Savoirs et compétences :}\\
\vspace{-.1cm}
\begin{itemize}[label=\ding{112},font=\color{ocre}]
\item SN.C3 : Utiliser les bibliothèques de calcul standard
\item SN.S1 : Bibliothèques logicielles
\item SN.S4 : Problème discret multidimensionnel linéaire. Méthode de Gauss avec recherche partielle du pivot.
\end{itemize}
}}


%Infos sur les supports
\def\xxtitreexo{Complexité}
\def\xxsourceexo{\hspace{.2cm} \footnotesize{\textbf{Sources : }
}}
%\def\xxtitreexo{Titre EXO}
%\def\xxsourceexo{\hspace{.2cm} \footnotesize{Source EXO}}


%---------------------------------------------------------------------------



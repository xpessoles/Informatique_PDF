\def\t{\textbackslash t}
\def\n{\textbackslash n}

\section*{Lecture d'un fichier et écriture dans un fichier}
	
\subsection*{Ecriture}

 - Pour créer un fichier en écriture, on écrira:

\begin{lstlisting}
	f=open(nom\de\_fichier,w)
\end{lstlisting}

- on utilisera la chaine de caractère \texttt{'\t'} pour une tabulation (un espace) et la chaine de caractère \texttt{'\n'} pour un passage à la ligne.

- On rajoutera les chaines de caractères avec \texttt{f.write} par exemple:

\begin{lstlisting}
	f.write(''blabla'')
\end{lstlisting}

- Il ne faudra pas oublier de fermer le fichier avec 

\begin{lstlisting}
	f.close()
\end{lstlisting}


\begin{exemple}
	\ \\[1mm]
	\texttt{f=open('essai.txt','a')}	\\
	\texttt{f.write(''3/2\t=1.5\n'')}\\
	\texttt{f.write(''3\n'')}\\
	\texttt{f.write(''2\n1\n'')}\\
	\texttt{f.write(''5\n'')}\\
	\texttt{f.close()}
\end{exemple}

- On rappelle que l'on peut concaténer des chaines de caractères avec l'opérateur “+”. Par exemple \texttt{''3/2\t=1.5\n''} équivaut à \texttt{''3/2''+''\t''+''=''+''1.5''+''\n''}.


\subsection*{Lecture}

- Pour ouvrir un fichier en lecture, on écrira:

\begin{lstlisting}
	f=open(nom\de\_fichier,r)
\end{lstlisting}

- On peut obtenir la chaine de caractère formée du contenu du fichier avec la méthode \texttt{read()}.

- On peut ensuite créer, grâce à la méthode \texttt{split}, casser la chaine de caractère et obtenir une liste contenant les diverses données.

\begin{lstlisting}
	c=f.read().split()
\end{lstlisting}

- Il ne faudra pas oublier de fermer le fichier avec 

\begin{lstlisting}
	f.close()
\end{lstlisting}

\begin{exemple}
	\ \\[1mm]
	\texttt{f=open('essai.txt','r')}	\\
	\texttt{c=f.read().split()}
\end{exemple}

renvoie \texttt{['3/2', '=', '1.5', '3', '2', '1', '5']}


\section*{Manipulation d'image à l'aide de fichiers}
	
	On cherche à représenter des  images. Pour simplifier, on ne cherchera
pas  à  représenter les  couleurs  mais  seulement  la luminosité  des
différents points de l'image.  

\medskip Pour cela, on découpe l'image, supposée
rectangulaire,  en carrés  de même  taille (appelés \emph{pixels}). On
remplace  alors chaque
carré  par un  entier  naturel indiquant  la  luminosité (moyenne)  de
l'image sur le carré considéré (la luminosité est donnée dans une
unité arbitraire, les valeurs allant de $0$ pour un carré noir à $N$
pour un carré blanc, où $N$ est une valeur arbitrairement fixée). 

On
obtient ainsi une matrice de $n\times p$ valeurs entières appartenant
à l'intervalle $[0, N]$.

\bigskip Pour lire et écrire une telle image dans un fichier, on peut utiliser
le format de données PGM (Portable GrayMap), dans sa version texte
(\textit{plain format}).

Ce format de données consiste à représenter une image de la façon
suivante par un fichier.
\begin{itemize}
	\item[\textbullet] Le fichier est un fichier texte (ne comportant que des
	caractères ASCII).
	\item[\textbullet] On appellera \emph{blanc} tout caractère qui est soit un retour à la
	ligne, soit un caractère espace, soit un caractère de tabulation (en
	fait, un autre caractère)
	\item[\textbullet] Toutes les valeurs contenues dans le fichier sont séparées par
	un ou plusieurs blancs.\footnote{À toutes fins utiles, pour toute
		chaîne de caractères \texttt{s}, l'expression \texttt{s.split()} désigne
		la liste obtenue par découpage de \texttt{s} en utilisant les blancs
		comme séparateurs.}
	\item[\textbullet] La première ligne du fichier doit contenir la «valeur magique»
	constituée des deux caractères «P2» (cette contrainte sert à
	distinguer un fichier pgm en niveaux de gris d'un autre type de fichier) et doit être
	suivie d'un blanc.
	\item[\textbullet] Les autres valeurs écrites dans le fichier sont toutes des
	entiers naturels, écrits en base $10$ (autrement
	dit, «douze» est représenté par la succession des caractères $1$ et $2$).
	\item[\textbullet] Après  la valeur  magique, on trouve  un nombre  représentant la
	largeur $p$ de la matrice puis un nombre représentant la hauteur $n$
	de l'image,  puis un  nombre donnant la  valeur de $N$  choisie pour
	cette image  (intensité de gris représentant le  blanc), puis toutes
	les valeurs de  la matrice représentant l'image, dans  l'ordre où on
	les  lirait  normalement si  la  matrice  était  un texte  écrit  en
	français (c'est-à-dire de gauche à droite puis de haut en bas).
	\medskip Au final, un fichier d'extension pgm sera de la forme:
			
			\ \hskip5cm P2	
			
			\ \hskip5cm p \ n \ N 	
			
			\ \hskip5cm valeur1 
			
			\ \hskip5cm valeur2
			
			\ \hskip5cm valeur 3
			
			\ \hskip5.5cm $\vdots$
			
			\ \hskip5cm valeur n*p
	
\medskip 	
	\item[\textbullet] Le fichier peut contenir des commentaires ; ceux-ci commencent par le
	caractère \texttt{\#} et finissent avec le retour à la ligne
	suivant. Ils doivent être simplement ignorés. Pour faciliter le
	travail demandé par la suite, \emph{on supposera qu'il n'y a aucun
		commentaire dans les fichiers que l'on
		traitera}.
	\item[\textbullet] On doit avoir $N\in\ii{0, 2^{16}}$
	\item[\textbullet] Les lignes du fichier doivent faire au plus $70$ caractères.
\end{itemize}

\bigskip On rappelle qu'on peut convertir une chaîne représentant un nombre
décimal avec \texttt{int} et réciproquement, convertir un nombre en chaîne
le représentant avec \texttt{str}.

\bigskip Enfin, on représentera une matrice (de nombres) à $n$ lignes et $p$ colonnes comme un tableau (ou liste Python) de longueur $n$, chacun de ses éléments étant un tableau de longueur $p$. On décrit donc la matrice ligne par ligne. 
Ainsi, l'indice \og ligne \texttt{i} colonne \texttt{j} \fg\ de la matrice représentée par \texttt{M} sera \texttt{M[i][j]}.
\begin{exemple}
	La matrice $\begin{pmatrix} 0&1&2 \\ 3&4&5 \end{pmatrix}$ sera représentée par \texttt{[[0,1,2],[3,4,5]]}.
\end{exemple}

\section*{Travail demandé}

Commencez par recopier les fonctions  \texttt{image\_noire}, \texttt{dim} et  \texttt{lit\_valeurs} données ci dessous : ce sont des exemples utiles, utilisez les ! 

\begin{lstlisting}
	def image_noire(n, p):
	"""Construit la matrice n*p d'une image noire."""
	img = [0]*n
	for i in range(n):
	img[i] = [0]*p
	return img
	
	def dim(img):
	"""Donne le couple (n, p) des dimensions de la matrice img. n :
	nombre de lignes, p : nombre de colonnes. La matrice est supposée
	avoir au moins une ligne."""
	n = len(img)
	p = len(img[0])
	return (n,p)
	
\end{lstlisting}

\eject \begin{lstlisting}
	
	def lit_valeurs(nom_de_fichier):
	"""Lit le contenu du fichier image f et renvoie la liste des
	valeurs lues (séparées par des blancs) sous forme d'une liste
	de chaines de caractères. La première valeur est normalement
	'P2'."""
	with open(nom_de_fichier, 'r') as f:
	c = f.read()
	return c.split()
\end{lstlisting}


N'hésitez pas à ouvrir l'image \texttt{degrade.pgm} et \texttt{joconde.pgm} dans un éditeur de texte puis dans un lecteur d'images, afin de comprendre le codage des images.


\subsection*{Sauvegarde d'images}

\question{ Écrire une fonction \texttt{sauve\_image(img, N, nom\_de\_fichier)} prenant en argument
	une matrice \texttt{img} représentant une image, l'entier \texttt{N} comme
	niveau de gris maximal (dans $\ii{0,2^{16}}$) ainsi qu'une chaîne \texttt{nom\_de\_fichier} et
	sauvant l'image dans le fichier nommé \texttt{nom\_de\_fichier}, au format PGM.}
	
	\question{\ Écrire une fonction \texttt{sauve\_rectangle\_noir(n, p, N, nom\_de\_fichier)}
	sauvant dans le fichier \texttt{nom\_de\_fichier} un rectangle noir, de côté
	$n$ pixels de hauteur et $p$ de largeur, où le blanc est
	d'intensité $N$.}
	
 	Vérifier que l'image produite par
	\begin{lstlisting}
		sauve_rectangle_noir(100, 200, 255, 'rectangle_noir.pgm')
	\end{lstlisting}
	est bien ce que   vous attendiez grâce à GIMP ou la visionneuse d'images.
	
	\question{\ Écrire une fonction
	\texttt{sauve\_rectangle\_blanc(n, p, N, nom\_de\_fichier)}
	sauvant dans le fichier   \texttt{nom\_de\_fichier} un rectangle blanc, de côté $n$ pixels de hauteur et $p$ de
	largeur, où le blanc est d'intensité $N$.}

De même, vérifiez l'image
	produite par
	\begin{lstlisting}
		sauve_rectangle_blanc(100, 200, 255, 'rectangle_blanc.pgm'). 
	\end{lstlisting}}
	
	\question{\ (Question facultative) Écrire une fonction \texttt{sauve\_echiquier(p, N, nom\_de\_fichier)} produisant
	dans le fichier \texttt{nom\_de\_fichier} l'image d'un échiquier, où chaque case de
	l'échiquier a pour côté $p$ pixels et $N$ est le niveau d'intensité
	du blanc. Pour mémoire un échiquier a 64 cases, et dans sa
	représentation traditionnelle, la case en bas à droite est blanche.}
	
	
	\subsection*{Lecture et modification d'images}
	
	On pourra, pour tester les fonctions écrites ici, utiliser d'une part
	les images précédemment produites, d'autre part utiliser l'image
	disponible sur le site de la classe.
	
	\question{\ Écrire une fonction \texttt{lit\_image(nom\_de\_fichier)}, où
	la chaîne \texttt{nom\_de\_fichier} représente le nom  d'un fichier PGM et renvoyant un couple \texttt{(img, N)} où \texttt{N} est
	le niveau d'intensité du blanc de l'image et \texttt{img} la matrice des
	pixels.} \\ \\
	\emph{On suppose que le fichier respecte les contraintes données
		dans l'énoncé et on ne fera aucun effort particulier pour gérer les
		situations où il ne les respecterait pas. \\
		Par exemple, votre
		fonction a le droit d'accepter un fichier dont les lignes
		font plus de $70$ caractères. On suppose de plus que le fichier ne
		contient aucun commentaire.}\\

\medskip

\textbf{Remarque:}
		On pourra utiliser la fonction \texttt{lit\_valeurs}, que vous trouverez sur le site de classe.
 
 \bigskip
 
 
	\question{\ Écrire une fonction \texttt{negatif(fichier\_entree, fichier\_sortie)} prenant en argument
	deux noms de fichiers \texttt{fichier\_entree} et \texttt{fichier\_sortie}. La fonction lit d'abord le fichier
	\texttt{fichier\_entree} et crée le fichier \texttt{fichier\_sortie} obtenu en remplaçant chaque pixel
	de niveau de gris $k$ par un pixel de niveau $N-k$, où $N$ est l'intensité du blanc du fichier d'entrée.}
	
	\question{\ (Question facultative) Écrire une fonction \texttt{rotation90(fichier\_entree, fichier\_sortie)}
	lisant le fichier 
	\texttt{fichier\_entree} et créant le fichier \texttt{fichier\_sortie} obtenu en effectuant une rotation
	de $90$ degrés (dans le sens trigonométrique) de l'image originale.}
	

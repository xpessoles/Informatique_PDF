% Déclaration des titres
% -------------------------------------


\graphicspath{{../../../style/png/}{images/}{../../../Informatique/Exercices/S1_05_FonctionsRecursives/fiche/}{../../../Informatique/Exercices/S1_05_FonctionsRecursives/10_factoriel/}{../../../Informatique/Exercices/S1_05_FonctionsRecursives/11_RechercheDichotomique_recursive/}{../../../Informatique/Exercices/S1_05_FonctionsRecursives/08_palindrome/}{../../../Informatique/Exercices/S1_05_FonctionsRecursives/05_BubbleBobble/}}
\lstinputpath{{../../../Informatique/Exercices/S1_05_FonctionsRecursives/fiche/}{../../../Informatique/Exercices/S1_05_FonctionsRecursives/10_factoriel/}{../../../Informatique/Exercices/S1_05_FonctionsRecursives/11_RechercheDichotomique_recursive/}{../../../Informatique/Exercices/S1_05_FonctionsRecursives/08_palindrome/}{../../../Informatique/Exercices/S1_05_FonctionsRecursives/05_BubbleBobble/}}

\def\discipline{Informatique}
\def\xxtete{Informatique}

\def\classe{\textsf{MPSI}}
\def\xxnumpartie{Semestre 1}
\def\xxpartie{}
\def\xxdate{26 Janvier 2023}

\def\xxchapitre{2}
\def\xxnumchapitre{2}
\def\xxnomchapitre{Fonctions récursives}
\def\xxnumactivite{08}

\def\xxposongletx{2}
\def\xxposonglettext{1.45}
\def\xxposonglety{19}%16

\def\xxonglet{\textsf{Cycle 01}}
\def\xxauteur{\textsl{E. Durif -- X. Pessoles \\ J.-P. Berne }}


\def\xxpied{%
\xxnumpartie -- \xxnomchapitre \\
%Cycle \xxnumpartie -- \xxpartie\\
%Chapitre \xxnumchapitre -- \xxactivite -\xxnumactivite -- \xxnomchapitre%
%Chapitre \xxnumchapitre -- 
\xxactivite \xxnumactivite %
}

\setcounter{secnumdepth}{5}
\chapterimage{Fond_ANA}
\def\xxfigures{}

\def\xxcompetences{%
\textsl{%
%\vspace{-.5cm}
\textbf{Savoirs et compétences :}\\
\vspace{-.1cm}
\begin{itemize}[label=\ding{112},font=\color{ocre}]
\item AN.C3 : S'interroger sur l'efficacité algorithmique temporelle d'un algorithme
\item AN.S1 : Recherche dans une liste, recherche du maximum dans une liste de nombres, calcul de la moyenne et de la variance.
\item AN.S2 : Recherche par dichotomie. 
\item AN.S4 : Recherche d'un mot dans une chaîne de caractères. 
\end{itemize}
}}


%Infos sur les supports
\def\xxtitreexo{Fonctions récursives}
\def\xxsourceexo{\hspace{.2cm} \footnotesize{\textbf{Sources : }



exercice 5 : Jean-Pierre Becirspahic
}}
%\def\xxtitreexo{Titre EXO}
%\def\xxsourceexo{\hspace{.2cm} \footnotesize{Source EXO}}


%---------------------------------------------------------------------------



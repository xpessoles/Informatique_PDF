
	
\begin{Large}
	Proposition de corrigé du TP 3
\end{Large}



\section*{Recherche dans un tableau}

\lstinputlisting[linerange={3-44},scale=0.75]{TP03corr.py}

On obtient pour chaque $i$ compris entre 0 et $n-1$, $n-i-1$ comparaisons.

Au total, on en a $n**2-n(n-1)/2-n$ lequel est equivalent a $n**2/2$.

D'où une complexité quadratique.


\eject

\section*{Recherche d'un mot dans un texte}

\lstinputlisting[linerange={59-89},scale=0.75]{TP03corr.py}

\eject \section*{Tri à bulles}

\lstinputlisting[linerange={93-132},scale=0.75]{TP03corr.py}

La première boucle $k$ ex\'ecute $n-1$ opérations. Pour chaque $k$, il y a $n-k$ boucles $i$ et une comparaison pour chacune. Au total, il y a une complexite de $\displaystyle\sum_{k=1}^{n-1}\sum_{i=0}^{n-k} 1 =n(n-1)/2 = O(n**2)$:
On obtient une complexité quadratique.



\eject 
\lstinputlisting[linerange={137-149},scale=0.75]{TP03corr.py}

Cette fonction est a privilégier lorsque le tableau contient des éléments minima en fin de tableau, notamment lorsque le tableau est décroissant par exemple.

Néanmoins, dans le pire des cas, le nombre de comparaison est la somme sur $k$ variant de 1 à $(n-1)//2$ de $(n-k-1)-(k-1)+1+(n-k-1)-(k-1)+1=2(n-2k+1)$. Cela donne encore une complexité quadratique car équivalent à un terme de la forme $a.n**2$.



\end{document}
% Déclaration des titres
% -------------------------------------


\graphicspath{{../../../style/png/}{images/}{../../../Informatique/exercices//}}
\lstinputpath{{../../../Informatique/exercices//}}

\def\discipline{Informatique}
\def\xxtete{Informatique}

\def\classe{\textsf{MPSI}}
\def\xxnumpartie{1}
\def\xxpartie{}
\def\xxdate{9 Septembre 2021}

\def\xxchapitre{1}
\def\xxnumchapitre{1}
\def\xxnomchapitre{Recherche séquentielle}
\def\xxnumactivite{01}

\def\xxposongletx{2}
\def\xxposonglettext{1.45}
\def\xxposonglety{19}%16

\def\xxonglet{\textsf{Cycle 01}}
\def\xxauteur{\textsl{E. Durif -- X. Pessoles \\ J.P. Berne }}


\def\xxpied{%
Cycle \xxnumpartie -- \xxpartie\\
Chapitre \xxnumchapitre -- \xxactivite -\xxnumactivite -- \xxnomchapitre%
}

\setcounter{secnumdepth}{5}
\chapterimage{Fond_ALG}
\def\xxfigures{}

\def\xxcompetences{%
\textsl{%
%\vspace{-.5cm}
\textbf{Savoirs et compétences :}\\
\vspace{-.1cm}
\begin{itemize}[label=\ding{112},font=\color{ocre}]
\item AA.C01 : Manipuler un OS ou un IDE
\item AA.S01 : Se familiariser aux principaux composants d'une machine numérique
\item AA.S2 : Se familiariser à la manipulation d'un OS
\item AA.S3 : Se familiariser à la manipulation d'un IDE
\end{itemize}
}}


%Infos sur les supports
\def\xxtitreexo{Recherche séquentielle}
\def\xxsourceexo{\hspace{.2cm} \footnotesize{\textbf{Sources : }
}}
%\def\xxtitreexo{Titre EXO}
%\def\xxsourceexo{\hspace{.2cm} \footnotesize{Source EXO}}


%---------------------------------------------------------------------------



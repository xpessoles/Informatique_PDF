% Déclaration des titres
% -------------------------------------


\graphicspath{{../../../style/png/}{images/}{../../../Informatique/exercices/11_plot/generalites/}}
\lstinputpath{{../../../Informatique/exercices/11_plot/generalites/}}

\def\discipline{Informatique}
\def\xxtete{Informatique}

\def\classe{\textsf{MPSI}}
\def\xxnumpartie{Semestre 1}
\def\xxpartie{}
\def\xxdate{10 Mars 2022}

\def\xxchapitre{2}
\def\xxnumchapitre{2}
\def\xxnomchapitre{TP sur les 5 premiers itemes du S2}
\def\xxnumactivite{13}

\def\xxposongletx{2}
\def\xxposonglettext{1.45}
\def\xxposonglety{19}%16

\def\xxonglet{\textsf{Cycle 01}}
\def\xxauteur{\textsl{E. Durif -- X. Pessoles \\ J.P. Berne }}


\def\xxpied{%
Cycle \xxnumpartie -- \xxpartie\\
Chapitre \xxnumchapitre -- \xxactivite -\xxnumactivite -- \xxnomchapitre%
}

\setcounter{secnumdepth}{5}
\chapterimage{}
\def\xxfigures{}

\def\xxcompetences{%
\textsl{%
%\vspace{-.5cm}
\textbf{Savoirs et compétences :}\\
\vspace{-.1cm}
\begin{itemize}[label=\ding{112},font=\color{ocre}]
\item SN.C2 : Étudier l'effet d'une variation des paramètres sur le temps de calcul, sur la précision des résultats, sur la forme des solutions pour des programmes d'ingénierie numérique choisis, tout en contextualisant l'observation du temps de calcul par rapport à la complexité algorithmique de ces programmes
\item SN.S3 : Problème dynamique à une dimension,  linéaire ou non. Méthode d'Euler.
\end{itemize}
}}


%Infos sur les supports
\def\xxtitreexo{TP sur les 5 premiers itemes du S2}
\def\xxsourceexo{\hspace{.2cm} \footnotesize{\textbf{Sources : }
}}
%\def\xxtitreexo{Titre EXO}
%\def\xxsourceexo{\hspace{.2cm} \footnotesize{Source EXO}}


%---------------------------------------------------------------------------



% Déclaration des titres
% -------------------------------------


\graphicspath{{../../../style/png/}{images/}{../../../Informatique/Exercices/S1_07_Images/01_TraitementPGM/}}
\lstinputpath{{../../../Informatique/Exercices/S1_07_Images/01_TraitementPGM/}}

\def\discipline{Informatique}
\def\xxtete{Informatique}

\def\classe{\textsf{MPSI}}
\def\xxnumpartie{Semestre 1}
\def\xxpartie{}
\def\xxdate{30 Mars 2023}

\def\xxchapitre{3}
\def\xxnumchapitre{3}
\def\xxnomchapitre{Traitement d'image}
\def\xxnumactivite{14}

\def\xxposongletx{2}
\def\xxposonglettext{1.45}
\def\xxposonglety{19}%16

\def\xxonglet{\textsf{Cycle 01}}
\def\xxauteur{\textsl{E. Durif -- X. Pessoles \\ J.-P. Berne }}


\def\xxpied{%
\xxnumpartie -- \xxnomchapitre \\
%Cycle \xxnumpartie -- \xxpartie\\
%Chapitre \xxnumchapitre -- \xxactivite -\xxnumactivite -- \xxnomchapitre%
%Chapitre \xxnumchapitre -- 
\xxactivite \xxnumactivite %
}

\setcounter{secnumdepth}{5}
\chapterimage{Fond_SIMU}
\def\xxfigures{}

\def\xxcompetences{%
\textsl{%
%\vspace{-.5cm}
\textbf{Savoirs et compétences :}\\
\vspace{-.1cm}
\begin{itemize}[label=\ding{112},font=\color{ocre}]
\item Th. 3 : Utilisation de modules, de bibliothèques.
\item AA.S9 : Instructions itératives
\item SN.C1 : Réaliser un programme complet structuré
\item SN.C2 : Étudier l'effet d'une variation des paramètres sur le temps de calcul, sur la précision des résultats, sur la forme des solutions pour des programmes d'ingénierie numérique choisis, tout en contextualisant l'observation du temps de calcul par rapport à la complexité algorithmique de ces programmes
\item SN.C3 : Utiliser les bibliothèques de calcul standard
\item SN.C4 : Utiliser les bibliothèques standard pour afficher les résultats sous forme graphique
\item SN.C5 : Tenir compte des aspects pratiques comme l'impact des erreurs d'arrondi sur les résultats, le temps de calcul ou le stockage en mémoire.
\end{itemize}
}}


%Infos sur les supports
\def\xxtitreexo{Traitement d'image}
\def\xxsourceexo{\hspace{.2cm} \footnotesize{\textbf{Sources : }
}}
%\def\xxtitreexo{Titre EXO}
%\def\xxsourceexo{\hspace{.2cm} \footnotesize{Source EXO}}


%---------------------------------------------------------------------------



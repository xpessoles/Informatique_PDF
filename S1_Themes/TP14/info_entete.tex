% Déclaration des titres
% -------------------------------------


\graphicspath{{../../../style/png/}{images/}{../../../Informatique/exercices/11_plot/generalites/}}
\lstinputpath{{../../../Informatique/exercices/11_plot/generalites/}}

\def\discipline{Informatique}
\def\xxtete{Informatique}

\def\classe{\textsf{MPSI}}
\def\xxnumpartie{Semestre 1}
\def\xxpartie{}
\def\xxdate{24 Mars 2022}

\def\xxchapitre{3}
\def\xxnumchapitre{3}
\def\xxnomchapitre{Invariants de boucles}
\def\xxnumactivite{14}

\def\xxposongletx{2}
\def\xxposonglettext{1.45}
\def\xxposonglety{19}%16

\def\xxonglet{\textsf{Cycle 01}}
\def\xxauteur{\textsl{E. Durif -- X. Pessoles \\ J.-P. Berne }}


\def\xxpied{%
\xxnumpartie -- \xxnomchapitre \\
%Cycle \xxnumpartie -- \xxpartie\\
%Chapitre \xxnumchapitre -- \xxactivite -\xxnumactivite -- \xxnomchapitre%
%Chapitre \xxnumchapitre -- 
\xxactivite \xxnumactivite %
}

\setcounter{secnumdepth}{5}
\chapterimage{}
\def\xxfigures{}

\def\xxcompetences{%
\textsl{%
%\vspace{-.5cm}
\textbf{Savoirs et compétences :}\\
\vspace{-.1cm}
\begin{itemize}[label=\ding{112},font=\color{ocre}]
\item SN.C2 : Étudier l'effet d'une variation des paramètres sur le temps de calcul, sur la précision des résultats, sur la forme des solutions pour des programmes d'ingénierie numérique choisis, tout en contextualisant l'observation du temps de calcul par rapport à la complexité algorithmique de ces programmes
\item SN.C3 : Utiliser les bibliothèques de calcul standard
\item SN.C4 : Utiliser les bibliothèques standard pour afficher les résultats sous forme graphique
\item SN.C5 : Tenir compte des aspects pratiques comme l'impact des erreurs d'arrondi sur les résultats, le temps de calcul ou le stockage en mémoire.
\item SN.S1 : Bibliothèques logicielles
\item SN.S2 : Problème stationnaire à une dimension. Méthode de dichotomie, méthode de Newton.
\end{itemize}
}}


%Infos sur les supports
\def\xxtitreexo{Invariants de boucles}
\def\xxsourceexo{\hspace{.2cm} \footnotesize{\textbf{Sources : }
}}
%\def\xxtitreexo{Titre EXO}
%\def\xxsourceexo{\hspace{.2cm} \footnotesize{Source EXO}}


%---------------------------------------------------------------------------



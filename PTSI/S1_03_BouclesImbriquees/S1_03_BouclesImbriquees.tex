\fichetrue
\proftrue
\tdfalse
\coursfalse

\def\xxnumchapitre{Révisions 1 \vspace{.2cm}}
\def\xxchapitre{\hspace{.12cm} Définitions préliminaires et détermination des performances}
\def\xxYCartouche{-2.25cm}
\def\xxposongletx{2}
\def\xxposonglettext{1.45}
\def\xxposonglety{19}%16

\def\xxonglet{Cy 01 -- Rév 1}

\def\xxactivite{Fiche}


\def\xxpied{%
Cycle 01 -- Modéliser le comportement des systèmes multiphysiques\\
Révision 1 -- \xxactivite%
}

\setcounter{secnumdepth}{5}
%---------------------------------------------------------------------------

\input{\repStyle/new_pagegarde}
%\iflivret
%\input{../../style/new_pagegarde}
%\else
%\input{../../style/new_pagegarde}
%\fi
\vspace{1.5cm}
\pagestyle{fancy}
\thispagestyle{plain}
\setcounter{section}{0}



Thème : Algorithmes opérant sur une structure séquentielle par boucles imbriquées. 
Commentaires :
\begin{itemize}
\item recherche d'un facteur dans un texte;
\item recherche des deux valeurs les plus proches dans un tableau;
\item tri à bulles;
\item notion de complexité quadratique
\item outils pour valider la correction de l'algorithme
\end{itemize}


\section{Parcours d'une liste de listes}
Les listes de listes permettent de mettre les données en deux dimensions. 
\begin{exemple}
~\\
\begin{multicols}{3}
Grille de mots mêlés.
\begin{center}
\begin{tabular}{|c|c|c|}
\hline
L & E & S \\ \hline
E & T & E \\ \hline
S & E & C \\ \hline
\end{tabular}
\end{center}

\begin{lstlisting}
grille = [['L','E','S'],['E','T','E'],['S','E','C']]
\end{lstlisting}

\vfill\null
\columnbreak

Table de multiplication
\begin{center}
\begin{tabular}{|c||c|c|c|}
\hline
$\times $ & 1 & 2 & 3 \\
\hline
\hline
1 & 1 & 2 & 3 \\
2 & 2 & 4 &  6 \\
3 & 3 & 6 &  9 \\
\hline
\end{tabular}
\end{center}

\begin{lstlisting}
table = [[1,2,3],[2,4,6],[3,6,9]]
\end{lstlisting}

\vfill\null
\columnbreak

Température en fonction du temps.

\begin{center}
\begin{tabular}{|c|c|}
\hline
T (s)& T°C \\
\hline \hline
1 & 18 \\ \hline
2 & 19 \\ \hline
3 & 21 \\ \hline
4  & 24 \\ \hline
\end{tabular}
\end{center}

\begin{lstlisting}
data = [[1,18], [2,19],[3,21],[4,24]]
\end{lstlisting}



\end{multicols}
\end{exemple}

\section{Parcours d'une sous-chaîne de caractères}


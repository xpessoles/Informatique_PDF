\documentclass[10pt,fleqn]{article} % Default font size and left-justified equations
\usepackage[%
    pdftitle={Introduction à l'algorithmique et à la programmation},
    pdfauthor={Xavier Pessoles}]{hyperref}

\newcommand{\repRel}{../../../..}
\input{\repRel/Style/packages}
\input{\repRel/Style/new_style}
\input{\repRel/Style/macros_SII}
\input{\repRel/Style/environment}


\newcommand{\macrocomp}{macro_competences}
\newcommand{\comp}{competences}
\newcommand{\td}{fichier_td}
\newcommand{\repExo}{dossier}
\newcommand{\repStyle}{\repRel/Style}

\def\xxYCartouche{-2.25cm}
\def\xxYongletGarde{.5cm}
\def\xxYOnget{.9cm}

\begin{document}

\livrettrue
\graphicspath{{\repStyle/png/}}
\input{\repStyle/Info_01_Entete}
\graphicspath{{\repStyle/png/}{\repRel/Informatique/Semestre_01/01_Bases/Cours/images/}}
%\input{\repRel/Informatique/Semestre_01/01_Bases/Cours/01_Bases.tex}


\def\xxcompetences{}
\def\xxfigures{}
\def\xxchapitre{}
\def\xxtitreexo{Algorithmes gloutons}
\def\xxsourceexo{}
\def\xxactivite{TP 08 \ifprof  -- Corrigé \else \fi}
\input{\repRel/Style/pagegarde_TD}


\proffalse
\vspace{5cm}
\pagestyle{fancy}
\thispagestyle{plain}

\graphicspath{{\repStyle/png/}{\repRel/Informatique/Exercices/S1_07_Images/01_TraitementPGM/}}
%\input{\repRel/Informatique/Exercices/S1_07_Images/01_TraitementPGM.tex}



\activite{Recherche d'un plus court chemin par un algorithme glouton}

Commencez par télécharger la trame du fichier \texttt{.py} sur le site de la classe : 
\url{https://mpsilamartin.github.io/info/TP/08_RechercheChemin.py}.

\renewcommand{\td}{06_RechercheChemin}
\renewcommand{\repExo}{/Informatique/Exercices/S1_06_Gloutons}
\graphicspath{{\repStyle/png/}{../../../../Informatique/Exercices/S1_06_Gloutons/06_RechercheChemin}}
\input{\repRel\repExo/\td}




\activite{Algorithme glouton du rendu de monnaie : facultatif}

\renewcommand{\td}{01_RenduMonnaie}
\renewcommand{\repExo}{/Informatique/Exercices/S1_06_Gloutons}
\graphicspath{{\repStyle/png/}{../../../../Informatique/Exercices/S1_06_Gloutons/01_RenduMonnaie}}
\input{\repRel\repExo/\td}





\end{document}




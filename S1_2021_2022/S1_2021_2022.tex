\documentclass[10pt,fleqn]{article} % Default font size and left-justified equations
\usepackage[%
    pdftitle={Exercices de SII},
    pdfauthor={Xavier Pessoles}]{hyperref}

\input{../../Style/packages}
\input{../../Style/new_style}
\input{../../Style/macros_SII}
\input{../../Style/environment}
\usetikzlibrary{patterns}
\usepackage{circuitikz}

\newcommand{\macrocomp}{macro_competences}
\newcommand{\comp}{competences}
\newcommand{\td}{fichier_td}
\newcommand{\repExo}{dossier}
\newcommand{\repStyle}{../../Style}

\def\xxYCartouche{-2.25cm}
\def\xxYongletGarde{.5cm}
\def\xxYOnget{.9cm}

\begin{document}

\def\xxcompetences{}
\def\xxfigures{}

\graphicspath{{\repStyle/png/}}

\setlength{\columnseprule}{.1pt}

%% MODELISER
% Page de garde

\input{../../Style/Entete_Themes_Info}
\input{../../Style/pagegarde_livret_exos}
\pagestyle{fancy}
\thispagestyle{plain}


\proffalse


%\section{Bases de Python}
%\documentclass[10pt,fleqn]{article} % Default font size and left-justified equations
\usepackage[%
    pdftitle={Modélisation dynamique},
    pdfauthor={Xavier Pessoles}]{hyperref}

\input{../../../style/packages}
\input{../../../style/new_style}
\input{../../../style/macros_SII}
\input{../../../style/environment}
% Déclaration des titres
% -------------------------------------


\graphicspath{{../../../style/png/}{images/}{../../../Informatique/exercices/consignes/consignes_tp1/}{../../../Informatique/exercices/20_architecture/ARCHI-002/}{../../../Informatique/exercices/01_python_bases/PYB-002-bis/}{../../../Informatique/exercices/01_python_bases/PYB-003-bis/}{../../../Informatique/exercices/01_python_bases/PYB-004/}}
\lstinputpath{{../../../Informatique/exercices/consignes/consignes_tp1/}{../../../Informatique/exercices/20_architecture/ARCHI-002/}{../../../Informatique/exercices/01_python_bases/PYB-002-bis/}{../../../Informatique/exercices/01_python_bases/PYB-003-bis/}{../../../Informatique/exercices/01_python_bases/PYB-004/}}

\def\discipline{Informatique}
\def\xxtete{Informatique}

\def\classe{\textsf{MPSI}}
\def\xxnumpartie{1}
\def\xxpartie{}
\def\xxdate{9 Septembre 2021}

\def\xxchapitre{1}
\def\xxnumchapitre{1}
\def\xxnomchapitre{Prise en main}
\def\xxnumactivite{01}

\def\xxposongletx{2}
\def\xxposonglettext{1.45}
\def\xxposonglety{19}%16

\def\xxonglet{\textsf{Cycle 01}}
\def\xxauteur{\textsl{E. Durif -- X. Pessoles \\ J.P. Berne }}


\def\xxpied{%
Cycle \xxnumpartie -- \xxpartie\\
Chapitre \xxnumchapitre -- \xxactivite -\xxnumactivite -- \xxnomchapitre%
}

\setcounter{secnumdepth}{5}
\chapterimage{Fond_ALG}
\def\xxfigures{}

\def\xxcompetences{%
\textsl{%
%\vspace{-.5cm}
\textbf{Savoirs et compétences :}\\
\vspace{-.1cm}
\begin{itemize}[label=\ding{112},font=\color{ocre}]
\item AA.C01 : Manipuler un OS ou un IDE
\item AA.S01 : Se familiariser aux principaux composants d'une machine numérique
\item AA.S2 : Se familiariser à la manipulation d'un OS
\item AA.S3 : Se familiariser à la manipulation d'un IDE
\end{itemize}
}}


%Infos sur les supports
\def\xxtitreexo{Prise en main}
\def\xxsourceexo{\hspace{.2cm} \footnotesize{\textbf{Sources : }




}}
%\def\xxtitreexo{Titre EXO}
%\def\xxsourceexo{\hspace{.2cm} \footnotesize{Source EXO}}


%---------------------------------------------------------------------------




\livretfalse





\begin{document}
% Sujet
%\TPtrue \fichefalse \proffalse \tdfalse \coursfalse \collefalse
%\corrigefalse
\graphicspath{{../../../style/png/}{images/}}


%%%% Paramétrage du cours %%%%
\def\xxactivite{Fiche}
\def\xxauteur{\textsl{É. Durif -- X. Pessoles -- J.-P. Berne}}
%\fichefalse
%\proftrue
%\tdfalse
%\courstrue



\def\xxYCartouche{-2.25cm}
\def\xxYongletGarde{.5cm}
\def\xxYOnget{.9cm}

% Déclaration des titres
% -------------------------------------


\graphicspath{{../../../style/png/}{images/}{../../../Informatique/exercices/consignes/consignes_tp1/}{../../../Informatique/exercices/20_architecture/ARCHI-002/}{../../../Informatique/exercices/01_python_bases/PYB-002-bis/}{../../../Informatique/exercices/01_python_bases/PYB-003-bis/}{../../../Informatique/exercices/01_python_bases/PYB-004/}}
\lstinputpath{{../../../Informatique/exercices/consignes/consignes_tp1/}{../../../Informatique/exercices/20_architecture/ARCHI-002/}{../../../Informatique/exercices/01_python_bases/PYB-002-bis/}{../../../Informatique/exercices/01_python_bases/PYB-003-bis/}{../../../Informatique/exercices/01_python_bases/PYB-004/}}

\def\discipline{Informatique}
\def\xxtete{Informatique}

\def\classe{\textsf{MPSI}}
\def\xxnumpartie{1}
\def\xxpartie{}
\def\xxdate{9 Septembre 2021}

\def\xxchapitre{1}
\def\xxnumchapitre{1}
\def\xxnomchapitre{Prise en main}
\def\xxnumactivite{01}

\def\xxposongletx{2}
\def\xxposonglettext{1.45}
\def\xxposonglety{19}%16

\def\xxonglet{\textsf{Cycle 01}}
\def\xxauteur{\textsl{E. Durif -- X. Pessoles \\ J.P. Berne }}


\def\xxpied{%
Cycle \xxnumpartie -- \xxpartie\\
Chapitre \xxnumchapitre -- \xxactivite -\xxnumactivite -- \xxnomchapitre%
}

\setcounter{secnumdepth}{5}
\chapterimage{Fond_ALG}
\def\xxfigures{}

\def\xxcompetences{%
\textsl{%
%\vspace{-.5cm}
\textbf{Savoirs et compétences :}\\
\vspace{-.1cm}
\begin{itemize}[label=\ding{112},font=\color{ocre}]
\item AA.C01 : Manipuler un OS ou un IDE
\item AA.S01 : Se familiariser aux principaux composants d'une machine numérique
\item AA.S2 : Se familiariser à la manipulation d'un OS
\item AA.S3 : Se familiariser à la manipulation d'un IDE
\end{itemize}
}}


%Infos sur les supports
\def\xxtitreexo{Prise en main}
\def\xxsourceexo{\hspace{.2cm} \footnotesize{\textbf{Sources : }




}}
%\def\xxtitreexo{Titre EXO}
%\def\xxsourceexo{\hspace{.2cm} \footnotesize{Source EXO}}


%---------------------------------------------------------------------------



\input{../../../Style/pagegarde_cours.tex}
%

\setlength{\columnseprule}{.1pt}

\pagestyle{fancy}
\thispagestyle{plain}


\vspace{3.5cm}

\def\columnseprulecolor{\color{ocre}}
\setlength{\columnseprule}{0.4pt} 

%%%%%%%%%%%%%%%%%%%%%%%


\vspace{3cm}
\documentclass[10pt,fleqn]{article} % Default font size and left-justified equations
\usepackage[%
    pdftitle={Modélisation dynamique},
    pdfauthor={Xavier Pessoles}]{hyperref}

\input{../../../style/packages}
\input{../../../style/new_style}
\input{../../../style/macros_SII}
\input{../../../style/environment}
% Déclaration des titres
% -------------------------------------


\graphicspath{{../../../style/png/}{images/}{../../../Informatique/exercices/consignes/consignes_tp1/}{../../../Informatique/exercices/20_architecture/ARCHI-002/}{../../../Informatique/exercices/01_python_bases/PYB-002-bis/}{../../../Informatique/exercices/01_python_bases/PYB-003-bis/}{../../../Informatique/exercices/01_python_bases/PYB-004/}}
\lstinputpath{{../../../Informatique/exercices/consignes/consignes_tp1/}{../../../Informatique/exercices/20_architecture/ARCHI-002/}{../../../Informatique/exercices/01_python_bases/PYB-002-bis/}{../../../Informatique/exercices/01_python_bases/PYB-003-bis/}{../../../Informatique/exercices/01_python_bases/PYB-004/}}

\def\discipline{Informatique}
\def\xxtete{Informatique}

\def\classe{\textsf{MPSI}}
\def\xxnumpartie{1}
\def\xxpartie{}
\def\xxdate{9 Septembre 2021}

\def\xxchapitre{1}
\def\xxnumchapitre{1}
\def\xxnomchapitre{Prise en main}
\def\xxnumactivite{01}

\def\xxposongletx{2}
\def\xxposonglettext{1.45}
\def\xxposonglety{19}%16

\def\xxonglet{\textsf{Cycle 01}}
\def\xxauteur{\textsl{E. Durif -- X. Pessoles \\ J.P. Berne }}


\def\xxpied{%
Cycle \xxnumpartie -- \xxpartie\\
Chapitre \xxnumchapitre -- \xxactivite -\xxnumactivite -- \xxnomchapitre%
}

\setcounter{secnumdepth}{5}
\chapterimage{Fond_ALG}
\def\xxfigures{}

\def\xxcompetences{%
\textsl{%
%\vspace{-.5cm}
\textbf{Savoirs et compétences :}\\
\vspace{-.1cm}
\begin{itemize}[label=\ding{112},font=\color{ocre}]
\item AA.C01 : Manipuler un OS ou un IDE
\item AA.S01 : Se familiariser aux principaux composants d'une machine numérique
\item AA.S2 : Se familiariser à la manipulation d'un OS
\item AA.S3 : Se familiariser à la manipulation d'un IDE
\end{itemize}
}}


%Infos sur les supports
\def\xxtitreexo{Prise en main}
\def\xxsourceexo{\hspace{.2cm} \footnotesize{\textbf{Sources : }




}}
%\def\xxtitreexo{Titre EXO}
%\def\xxsourceexo{\hspace{.2cm} \footnotesize{Source EXO}}


%---------------------------------------------------------------------------




\livretfalse





\begin{document}
% Sujet
%\TPtrue \fichefalse \proffalse \tdfalse \coursfalse \collefalse
%\corrigefalse
\graphicspath{{../../../style/png/}{images/}}


%%%% Paramétrage du cours %%%%
\def\xxactivite{Fiche}
\def\xxauteur{\textsl{É. Durif -- X. Pessoles -- J.-P. Berne}}
%\fichefalse
%\proftrue
%\tdfalse
%\courstrue



\def\xxYCartouche{-2.25cm}
\def\xxYongletGarde{.5cm}
\def\xxYOnget{.9cm}

% Déclaration des titres
% -------------------------------------


\graphicspath{{../../../style/png/}{images/}{../../../Informatique/exercices/consignes/consignes_tp1/}{../../../Informatique/exercices/20_architecture/ARCHI-002/}{../../../Informatique/exercices/01_python_bases/PYB-002-bis/}{../../../Informatique/exercices/01_python_bases/PYB-003-bis/}{../../../Informatique/exercices/01_python_bases/PYB-004/}}
\lstinputpath{{../../../Informatique/exercices/consignes/consignes_tp1/}{../../../Informatique/exercices/20_architecture/ARCHI-002/}{../../../Informatique/exercices/01_python_bases/PYB-002-bis/}{../../../Informatique/exercices/01_python_bases/PYB-003-bis/}{../../../Informatique/exercices/01_python_bases/PYB-004/}}

\def\discipline{Informatique}
\def\xxtete{Informatique}

\def\classe{\textsf{MPSI}}
\def\xxnumpartie{1}
\def\xxpartie{}
\def\xxdate{9 Septembre 2021}

\def\xxchapitre{1}
\def\xxnumchapitre{1}
\def\xxnomchapitre{Prise en main}
\def\xxnumactivite{01}

\def\xxposongletx{2}
\def\xxposonglettext{1.45}
\def\xxposonglety{19}%16

\def\xxonglet{\textsf{Cycle 01}}
\def\xxauteur{\textsl{E. Durif -- X. Pessoles \\ J.P. Berne }}


\def\xxpied{%
Cycle \xxnumpartie -- \xxpartie\\
Chapitre \xxnumchapitre -- \xxactivite -\xxnumactivite -- \xxnomchapitre%
}

\setcounter{secnumdepth}{5}
\chapterimage{Fond_ALG}
\def\xxfigures{}

\def\xxcompetences{%
\textsl{%
%\vspace{-.5cm}
\textbf{Savoirs et compétences :}\\
\vspace{-.1cm}
\begin{itemize}[label=\ding{112},font=\color{ocre}]
\item AA.C01 : Manipuler un OS ou un IDE
\item AA.S01 : Se familiariser aux principaux composants d'une machine numérique
\item AA.S2 : Se familiariser à la manipulation d'un OS
\item AA.S3 : Se familiariser à la manipulation d'un IDE
\end{itemize}
}}


%Infos sur les supports
\def\xxtitreexo{Prise en main}
\def\xxsourceexo{\hspace{.2cm} \footnotesize{\textbf{Sources : }




}}
%\def\xxtitreexo{Titre EXO}
%\def\xxsourceexo{\hspace{.2cm} \footnotesize{Source EXO}}


%---------------------------------------------------------------------------



\input{../../../Style/pagegarde_cours.tex}
%

\setlength{\columnseprule}{.1pt}

\pagestyle{fancy}
\thispagestyle{plain}


\vspace{3.5cm}

\def\columnseprulecolor{\color{ocre}}
\setlength{\columnseprule}{0.4pt} 

%%%%%%%%%%%%%%%%%%%%%%%


\vspace{3cm}
\documentclass[10pt,fleqn]{article} % Default font size and left-justified equations
\usepackage[%
    pdftitle={Modélisation dynamique},
    pdfauthor={Xavier Pessoles}]{hyperref}

\input{../../../style/packages}
\input{../../../style/new_style}
\input{../../../style/macros_SII}
\input{../../../style/environment}
\input{info_entete}

\livretfalse





\begin{document}
% Sujet
%\TPtrue \fichefalse \proffalse \tdfalse \coursfalse \collefalse
%\corrigefalse
\graphicspath{{../../../style/png/}{images/}}


%%%% Paramétrage du cours %%%%
\def\xxactivite{Fiche}
\def\xxauteur{\textsl{É. Durif -- X. Pessoles -- J.-P. Berne}}
%\fichefalse
%\proftrue
%\tdfalse
%\courstrue



\def\xxYCartouche{-2.25cm}
\def\xxYongletGarde{.5cm}
\def\xxYOnget{.9cm}

\input{info_entete}
\input{../../../Style/pagegarde_cours.tex}
%

\setlength{\columnseprule}{.1pt}

\pagestyle{fancy}
\thispagestyle{plain}


\vspace{3.5cm}

\def\columnseprulecolor{\color{ocre}}
\setlength{\columnseprule}{0.4pt} 

%%%%%%%%%%%%%%%%%%%%%%%


\vspace{3cm}
\input{../../../Informatique/Fiches/Fiche_01_Bases/Fiche_01_Bases}

\end{document}

\end{document}

\end{document}

\documentclass[10pt,fleqn]{article} % Default font size and left-justified equations
\usepackage[%
    pdftitle={Modélisation dynamique},
    pdfauthor={Xavier Pessoles}]{hyperref}

\input{../../../style/packages}
\input{../../../style/new_style}
\input{../../../style/macros_SII}
\input{../../../style/environment}
% Déclaration des titres
% -------------------------------------


\graphicspath{{../../../style/png/}{images/}{../../../Informatique/exercices/consignes/consignes_tp1/}{../../../Informatique/exercices/20_architecture/ARCHI-002/}{../../../Informatique/exercices/01_python_bases/PYB-002-bis/}{../../../Informatique/exercices/01_python_bases/PYB-003-bis/}{../../../Informatique/exercices/01_python_bases/PYB-004/}}
\lstinputpath{{../../../Informatique/exercices/consignes/consignes_tp1/}{../../../Informatique/exercices/20_architecture/ARCHI-002/}{../../../Informatique/exercices/01_python_bases/PYB-002-bis/}{../../../Informatique/exercices/01_python_bases/PYB-003-bis/}{../../../Informatique/exercices/01_python_bases/PYB-004/}}

\def\discipline{Informatique}
\def\xxtete{Informatique}

\def\classe{\textsf{MPSI}}
\def\xxnumpartie{1}
\def\xxpartie{}
\def\xxdate{9 Septembre 2021}

\def\xxchapitre{1}
\def\xxnumchapitre{1}
\def\xxnomchapitre{Prise en main}
\def\xxnumactivite{01}

\def\xxposongletx{2}
\def\xxposonglettext{1.45}
\def\xxposonglety{19}%16

\def\xxonglet{\textsf{Cycle 01}}
\def\xxauteur{\textsl{E. Durif -- X. Pessoles \\ J.P. Berne }}


\def\xxpied{%
Cycle \xxnumpartie -- \xxpartie\\
Chapitre \xxnumchapitre -- \xxactivite -\xxnumactivite -- \xxnomchapitre%
}

\setcounter{secnumdepth}{5}
\chapterimage{Fond_ALG}
\def\xxfigures{}

\def\xxcompetences{%
\textsl{%
%\vspace{-.5cm}
\textbf{Savoirs et compétences :}\\
\vspace{-.1cm}
\begin{itemize}[label=\ding{112},font=\color{ocre}]
\item AA.C01 : Manipuler un OS ou un IDE
\item AA.S01 : Se familiariser aux principaux composants d'une machine numérique
\item AA.S2 : Se familiariser à la manipulation d'un OS
\item AA.S3 : Se familiariser à la manipulation d'un IDE
\end{itemize}
}}


%Infos sur les supports
\def\xxtitreexo{Prise en main}
\def\xxsourceexo{\hspace{.2cm} \footnotesize{\textbf{Sources : }




}}
%\def\xxtitreexo{Titre EXO}
%\def\xxsourceexo{\hspace{.2cm} \footnotesize{Source EXO}}


%---------------------------------------------------------------------------




\livretfalse





\begin{document}
% Sujet
%\TPtrue \fichefalse \proffalse \tdfalse \coursfalse \collefalse
%\corrigefalse
\graphicspath{{../../../style/png/}{images/}{../../../Informatique/Fiches/Fiche_05_FonctionsRecursives/images/}}


%%%% Paramétrage du cours %%%%
\def\xxactivite{Fiche}
\def\xxauteur{\textsl{É. Durif -- X. Pessoles -- J.-P. Berne}}
%\fichefalse
%\proftrue
%\tdfalse
%\courstrue



\def\xxYCartouche{-2.25cm}
\def\xxYongletGarde{.5cm}
\def\xxYOnget{.9cm}

% Déclaration des titres
% -------------------------------------


\graphicspath{{../../../style/png/}{images/}{../../../Informatique/exercices/consignes/consignes_tp1/}{../../../Informatique/exercices/20_architecture/ARCHI-002/}{../../../Informatique/exercices/01_python_bases/PYB-002-bis/}{../../../Informatique/exercices/01_python_bases/PYB-003-bis/}{../../../Informatique/exercices/01_python_bases/PYB-004/}}
\lstinputpath{{../../../Informatique/exercices/consignes/consignes_tp1/}{../../../Informatique/exercices/20_architecture/ARCHI-002/}{../../../Informatique/exercices/01_python_bases/PYB-002-bis/}{../../../Informatique/exercices/01_python_bases/PYB-003-bis/}{../../../Informatique/exercices/01_python_bases/PYB-004/}}

\def\discipline{Informatique}
\def\xxtete{Informatique}

\def\classe{\textsf{MPSI}}
\def\xxnumpartie{1}
\def\xxpartie{}
\def\xxdate{9 Septembre 2021}

\def\xxchapitre{1}
\def\xxnumchapitre{1}
\def\xxnomchapitre{Prise en main}
\def\xxnumactivite{01}

\def\xxposongletx{2}
\def\xxposonglettext{1.45}
\def\xxposonglety{19}%16

\def\xxonglet{\textsf{Cycle 01}}
\def\xxauteur{\textsl{E. Durif -- X. Pessoles \\ J.P. Berne }}


\def\xxpied{%
Cycle \xxnumpartie -- \xxpartie\\
Chapitre \xxnumchapitre -- \xxactivite -\xxnumactivite -- \xxnomchapitre%
}

\setcounter{secnumdepth}{5}
\chapterimage{Fond_ALG}
\def\xxfigures{}

\def\xxcompetences{%
\textsl{%
%\vspace{-.5cm}
\textbf{Savoirs et compétences :}\\
\vspace{-.1cm}
\begin{itemize}[label=\ding{112},font=\color{ocre}]
\item AA.C01 : Manipuler un OS ou un IDE
\item AA.S01 : Se familiariser aux principaux composants d'une machine numérique
\item AA.S2 : Se familiariser à la manipulation d'un OS
\item AA.S3 : Se familiariser à la manipulation d'un IDE
\end{itemize}
}}


%Infos sur les supports
\def\xxtitreexo{Prise en main}
\def\xxsourceexo{\hspace{.2cm} \footnotesize{\textbf{Sources : }




}}
%\def\xxtitreexo{Titre EXO}
%\def\xxsourceexo{\hspace{.2cm} \footnotesize{Source EXO}}


%---------------------------------------------------------------------------



\input{../../../Style/pagegarde_cours.tex}
%

\setlength{\columnseprule}{.1pt}

\pagestyle{fancy}
\thispagestyle{plain}


\vspace{3.5cm}

\def\columnseprulecolor{\color{ocre}}
\setlength{\columnseprule}{0.4pt} 

%%%%%%%%%%%%%%%%%%%%%%%


\vspace{3cm}
\documentclass[10pt,fleqn]{article} % Default font size and left-justified equations
\usepackage[%
    pdftitle={Modélisation dynamique},
    pdfauthor={Xavier Pessoles}]{hyperref}

\input{../../../style/packages}
\input{../../../style/new_style}
\input{../../../style/macros_SII}
\input{../../../style/environment}
% Déclaration des titres
% -------------------------------------


\graphicspath{{../../../style/png/}{images/}{../../../Informatique/exercices/consignes/consignes_tp1/}{../../../Informatique/exercices/20_architecture/ARCHI-002/}{../../../Informatique/exercices/01_python_bases/PYB-002-bis/}{../../../Informatique/exercices/01_python_bases/PYB-003-bis/}{../../../Informatique/exercices/01_python_bases/PYB-004/}}
\lstinputpath{{../../../Informatique/exercices/consignes/consignes_tp1/}{../../../Informatique/exercices/20_architecture/ARCHI-002/}{../../../Informatique/exercices/01_python_bases/PYB-002-bis/}{../../../Informatique/exercices/01_python_bases/PYB-003-bis/}{../../../Informatique/exercices/01_python_bases/PYB-004/}}

\def\discipline{Informatique}
\def\xxtete{Informatique}

\def\classe{\textsf{MPSI}}
\def\xxnumpartie{1}
\def\xxpartie{}
\def\xxdate{9 Septembre 2021}

\def\xxchapitre{1}
\def\xxnumchapitre{1}
\def\xxnomchapitre{Prise en main}
\def\xxnumactivite{01}

\def\xxposongletx{2}
\def\xxposonglettext{1.45}
\def\xxposonglety{19}%16

\def\xxonglet{\textsf{Cycle 01}}
\def\xxauteur{\textsl{E. Durif -- X. Pessoles \\ J.P. Berne }}


\def\xxpied{%
Cycle \xxnumpartie -- \xxpartie\\
Chapitre \xxnumchapitre -- \xxactivite -\xxnumactivite -- \xxnomchapitre%
}

\setcounter{secnumdepth}{5}
\chapterimage{Fond_ALG}
\def\xxfigures{}

\def\xxcompetences{%
\textsl{%
%\vspace{-.5cm}
\textbf{Savoirs et compétences :}\\
\vspace{-.1cm}
\begin{itemize}[label=\ding{112},font=\color{ocre}]
\item AA.C01 : Manipuler un OS ou un IDE
\item AA.S01 : Se familiariser aux principaux composants d'une machine numérique
\item AA.S2 : Se familiariser à la manipulation d'un OS
\item AA.S3 : Se familiariser à la manipulation d'un IDE
\end{itemize}
}}


%Infos sur les supports
\def\xxtitreexo{Prise en main}
\def\xxsourceexo{\hspace{.2cm} \footnotesize{\textbf{Sources : }




}}
%\def\xxtitreexo{Titre EXO}
%\def\xxsourceexo{\hspace{.2cm} \footnotesize{Source EXO}}


%---------------------------------------------------------------------------




\livretfalse





\begin{document}
% Sujet
%\TPtrue \fichefalse \proffalse \tdfalse \coursfalse \collefalse
%\corrigefalse
\graphicspath{{../../../style/png/}{images/}{../../../Informatique/Fiches/Fiche_05_FonctionsRecursives/images/}}


%%%% Paramétrage du cours %%%%
\def\xxactivite{Fiche}
\def\xxauteur{\textsl{É. Durif -- X. Pessoles -- J.-P. Berne}}
%\fichefalse
%\proftrue
%\tdfalse
%\courstrue



\def\xxYCartouche{-2.25cm}
\def\xxYongletGarde{.5cm}
\def\xxYOnget{.9cm}

% Déclaration des titres
% -------------------------------------


\graphicspath{{../../../style/png/}{images/}{../../../Informatique/exercices/consignes/consignes_tp1/}{../../../Informatique/exercices/20_architecture/ARCHI-002/}{../../../Informatique/exercices/01_python_bases/PYB-002-bis/}{../../../Informatique/exercices/01_python_bases/PYB-003-bis/}{../../../Informatique/exercices/01_python_bases/PYB-004/}}
\lstinputpath{{../../../Informatique/exercices/consignes/consignes_tp1/}{../../../Informatique/exercices/20_architecture/ARCHI-002/}{../../../Informatique/exercices/01_python_bases/PYB-002-bis/}{../../../Informatique/exercices/01_python_bases/PYB-003-bis/}{../../../Informatique/exercices/01_python_bases/PYB-004/}}

\def\discipline{Informatique}
\def\xxtete{Informatique}

\def\classe{\textsf{MPSI}}
\def\xxnumpartie{1}
\def\xxpartie{}
\def\xxdate{9 Septembre 2021}

\def\xxchapitre{1}
\def\xxnumchapitre{1}
\def\xxnomchapitre{Prise en main}
\def\xxnumactivite{01}

\def\xxposongletx{2}
\def\xxposonglettext{1.45}
\def\xxposonglety{19}%16

\def\xxonglet{\textsf{Cycle 01}}
\def\xxauteur{\textsl{E. Durif -- X. Pessoles \\ J.P. Berne }}


\def\xxpied{%
Cycle \xxnumpartie -- \xxpartie\\
Chapitre \xxnumchapitre -- \xxactivite -\xxnumactivite -- \xxnomchapitre%
}

\setcounter{secnumdepth}{5}
\chapterimage{Fond_ALG}
\def\xxfigures{}

\def\xxcompetences{%
\textsl{%
%\vspace{-.5cm}
\textbf{Savoirs et compétences :}\\
\vspace{-.1cm}
\begin{itemize}[label=\ding{112},font=\color{ocre}]
\item AA.C01 : Manipuler un OS ou un IDE
\item AA.S01 : Se familiariser aux principaux composants d'une machine numérique
\item AA.S2 : Se familiariser à la manipulation d'un OS
\item AA.S3 : Se familiariser à la manipulation d'un IDE
\end{itemize}
}}


%Infos sur les supports
\def\xxtitreexo{Prise en main}
\def\xxsourceexo{\hspace{.2cm} \footnotesize{\textbf{Sources : }




}}
%\def\xxtitreexo{Titre EXO}
%\def\xxsourceexo{\hspace{.2cm} \footnotesize{Source EXO}}


%---------------------------------------------------------------------------



\input{../../../Style/pagegarde_cours.tex}
%

\setlength{\columnseprule}{.1pt}

\pagestyle{fancy}
\thispagestyle{plain}


\vspace{3.5cm}

\def\columnseprulecolor{\color{ocre}}
\setlength{\columnseprule}{0.4pt} 

%%%%%%%%%%%%%%%%%%%%%%%


\vspace{3cm}
\documentclass[10pt,fleqn]{article} % Default font size and left-justified equations
\usepackage[%
    pdftitle={Modélisation dynamique},
    pdfauthor={Xavier Pessoles}]{hyperref}

\input{../../../style/packages}
\input{../../../style/new_style}
\input{../../../style/macros_SII}
\input{../../../style/environment}
\input{info_entete}

\livretfalse





\begin{document}
% Sujet
%\TPtrue \fichefalse \proffalse \tdfalse \coursfalse \collefalse
%\corrigefalse
\graphicspath{{../../../style/png/}{images/}{../../../Informatique/Fiches/Fiche_05_FonctionsRecursives/images/}}


%%%% Paramétrage du cours %%%%
\def\xxactivite{Fiche}
\def\xxauteur{\textsl{É. Durif -- X. Pessoles -- J.-P. Berne}}
%\fichefalse
%\proftrue
%\tdfalse
%\courstrue



\def\xxYCartouche{-2.25cm}
\def\xxYongletGarde{.5cm}
\def\xxYOnget{.9cm}

\input{info_entete}
\input{../../../Style/pagegarde_cours.tex}
%

\setlength{\columnseprule}{.1pt}

\pagestyle{fancy}
\thispagestyle{plain}


\vspace{3.5cm}

\def\columnseprulecolor{\color{ocre}}
\setlength{\columnseprule}{0.4pt} 

%%%%%%%%%%%%%%%%%%%%%%%


\vspace{3cm}
\input{../../../Informatique/Fiches/Fiche_05_FonctionsRecursives/Fiche_05_FonctionsRecursives}

\end{document}

\end{document}

\end{document}

\begin{multicols}{2}
\end{multicols}

\newpage

%\begin{multicols}{2}
%\section{Algorithmique}
%\renewcommand{\repExo}{../../Informatique/Exercices/02_algo} 
 
\renewcommand{\td}{ALG-000}
\graphicspath{{\repStyle/png/}{\repExo/\td/}}
\input{\repExo/\td.tex} 
 
\renewcommand{\td}{ALG-001}
\graphicspath{{\repStyle/png/}{\repExo/\td/}}
\input{\repExo/\td.tex} 
 
\renewcommand{\td}{ALG-002}
\graphicspath{{\repStyle/png/}{\repExo/\td/}}
\input{\repExo/\td.tex} 
 
\renewcommand{\td}{ALG-003}
\graphicspath{{\repStyle/png/}{\repExo/\td/}}
\input{\repExo/\td.tex} 
 
\renewcommand{\td}{ALG-004}
\graphicspath{{\repStyle/png/}{\repExo/\td/}}
\input{\repExo/\td.tex} 
 
\renewcommand{\td}{ALG-005}
\graphicspath{{\repStyle/png/}{\repExo/\td/}}
\input{\repExo/\td.tex} 
 
\renewcommand{\td}{ALG-006}
\graphicspath{{\repStyle/png/}{\repExo/\td/}}
\input{\repExo/\td.tex} 
 
\renewcommand{\td}{ALG-007}
\graphicspath{{\repStyle/png/}{\repExo/\td/}}
\input{\repExo/\td.tex} 
 
\renewcommand{\td}{ALG-008}
\graphicspath{{\repStyle/png/}{\repExo/\td/}}
\input{\repExo/\td.tex} 
 
\renewcommand{\td}{ALG-009}
\graphicspath{{\repStyle/png/}{\repExo/\td/}}
\input{\repExo/\td.tex} 
 
\renewcommand{\td}{ALG-010}
\graphicspath{{\repStyle/png/}{\repExo/\td/}}
\input{\repExo/\td.tex} 
 
\renewcommand{\td}{ALG-011}
\graphicspath{{\repStyle/png/}{\repExo/\td/}}
\input{\repExo/\td.tex} 
 
\renewcommand{\td}{ALG-012}
\graphicspath{{\repStyle/png/}{\repExo/\td/}}
\input{\repExo/\td.tex} 
 
\renewcommand{\td}{ALG-013}
\graphicspath{{\repStyle/png/}{\repExo/\td/}}
\input{\repExo/\td.tex} 
 
\renewcommand{\td}{ALG-014}
\graphicspath{{\repStyle/png/}{\repExo/\td/}}
\input{\repExo/\td.tex} 
 
\renewcommand{\td}{ALG-015}
\graphicspath{{\repStyle/png/}{\repExo/\td/}}
\input{\repExo/\td.tex} 
 
\renewcommand{\td}{ALG-016}
\graphicspath{{\repStyle/png/}{\repExo/\td/}}
\input{\repExo/\td.tex} 
 
\renewcommand{\td}{ALG-017}
\graphicspath{{\repStyle/png/}{\repExo/\td/}}
\input{\repExo/\td.tex} 
 
\renewcommand{\td}{ALG-018}
\graphicspath{{\repStyle/png/}{\repExo/\td/}}
\input{\repExo/\td.tex} 
 
\renewcommand{\td}{ALG-019}
\graphicspath{{\repStyle/png/}{\repExo/\td/}}
\input{\repExo/\td.tex} 
 
\renewcommand{\td}{ALG-020}
\graphicspath{{\repStyle/png/}{\repExo/\td/}}
\input{\repExo/\td.tex} 
 
\renewcommand{\td}{ALG-021}
\graphicspath{{\repStyle/png/}{\repExo/\td/}}
\input{\repExo/\td.tex} 
 
\renewcommand{\td}{ALG-022}
\graphicspath{{\repStyle/png/}{\repExo/\td/}}
\input{\repExo/\td.tex} 
 
\renewcommand{\td}{bode}
\graphicspath{{\repStyle/png/}{\repExo/\td/}}
\input{\repExo/\td.tex} 
 
\renewcommand{\td}{mission_insight}
\graphicspath{{\repStyle/png/}{\repExo/\td/}}
\input{\repExo/\td.tex} 
 

%\end{multicols}
%\newpage
%
%\begin{multicols}{2}
%\section{Tableaux}
%\renewcommand{\repExo}{../../Informatique/Exercices/03_tableaux} 
 
\renewcommand{\td}{carre_magique}
\graphicspath{{\repStyle/png/}{\repExo/\td/}}
\input{\repExo/\td.tex} 
 
\renewcommand{\td}{decryptage}
\graphicspath{{\repStyle/png/}{\repExo/\td/}}
\input{\repExo/\td.tex} 
 
\renewcommand{\td}{percolation}
\graphicspath{{\repStyle/png/}{\repExo/\td/}}
\input{\repExo/\td.tex} 

\renewcommand{\td}{TAB-000}
\graphicspath{{\repStyle/png/}{\repExo/\td/}}
\input{\repExo/\td.tex} 
 
\renewcommand{\td}{TAB-001}
\graphicspath{{\repStyle/png/}{\repExo/\td/}}
\input{\repExo/\td.tex} 
 
\renewcommand{\td}{TAB-002}
\graphicspath{{\repStyle/png/}{\repExo/\td/}}
\input{\repExo/\td.tex} 

\renewcommand{\td}{TAB-003}
\graphicspath{{\repStyle/png/}{\repExo/\td/}}
\input{\repExo/\td.tex} 
 
\renewcommand{\td}{TAB-004}
\graphicspath{{\repStyle/png/}{\repExo/\td/}}
\input{\repExo/\td.tex} 
 
\renewcommand{\td}{TAB-005}
\graphicspath{{\repStyle/png/}{\repExo/\td/}}
\input{\repExo/\td.tex} 
 
\renewcommand{\td}{TAB-006}
\graphicspath{{\repStyle/png/}{\repExo/\td/}}
\input{\repExo/\td.tex} 
 
\renewcommand{\td}{TAB-007}
\graphicspath{{\repStyle/png/}{\repExo/\td/}}
\input{\repExo/\td.tex} 
 
\renewcommand{\td}{TAB-008}
\graphicspath{{\repStyle/png/}{\repExo/\td/}}
\input{\repExo/\td.tex} 
 
\renewcommand{\td}{TAB-009}
\graphicspath{{\repStyle/png/}{\repExo/\td/}}
\input{\repExo/\td.tex} 
 
\renewcommand{\td}{TAB-010}
\graphicspath{{\repStyle/png/}{\repExo/\td/}}
\input{\repExo/\td.tex} 
 
\renewcommand{\td}{TAB-011}
\graphicspath{{\repStyle/png/}{\repExo/\td/}}
\input{\repExo/\td.tex} 
 
\renewcommand{\td}{TAB-012}
\graphicspath{{\repStyle/png/}{\repExo/\td/}}
\input{\repExo/\td.tex} 
 
\renewcommand{\td}{TAB-013}
\graphicspath{{\repStyle/png/}{\repExo/\td/}}
\input{\repExo/\td.tex} 

\renewcommand{\td}{TAB-014}
\graphicspath{{\repStyle/png/}{\repExo/\td/}}
\input{\repExo/\td.tex} 
 
\renewcommand{\td}{TAB-015}
\graphicspath{{\repStyle/png/}{\repExo/\td/}}
\input{\repExo/\td.tex} 
 
\renewcommand{\td}{TAB-016}
\graphicspath{{\repStyle/png/}{\repExo/\td/}}
\input{\repExo/\td.tex} 
 
\renewcommand{\td}{TAB-017}
\graphicspath{{\repStyle/png/}{\repExo/\td/}}
\input{\repExo/\td.tex} 
 
\renewcommand{\td}{TAB-018}
\graphicspath{{\repStyle/png/}{\repExo/\td/}}
\input{\repExo/\td.tex} 
 
\renewcommand{\td}{TAB-019}
\graphicspath{{\repStyle/png/}{\repExo/\td/}}
\input{\repExo/\td.tex} 
 
\renewcommand{\td}{TAB-020}
\graphicspath{{\repStyle/png/}{\repExo/\td/}}
\input{\repExo/\td.tex} 
 
\renewcommand{\td}{TAB-021}
\graphicspath{{\repStyle/png/}{\repExo/\td/}}
\input{\repExo/\td.tex} 
 
\renewcommand{\td}{TAB-022}
\graphicspath{{\repStyle/png/}{\repExo/\td/}}
\input{\repExo/\td.tex} 
 
\renewcommand{\td}{TAB-023}
\graphicspath{{\repStyle/png/}{\repExo/\td/}}
\input{\repExo/\td.tex} 
 
\renewcommand{\td}{TAB-024}
\graphicspath{{\repStyle/png/}{\repExo/\td/}}
\input{\repExo/\td.tex} 
 
\renewcommand{\td}{TAB-025}
\graphicspath{{\repStyle/png/}{\repExo/\td/}}
\input{\repExo/\td.tex} 
 
\renewcommand{\td}{TAB-026}
\graphicspath{{\repStyle/png/}{\repExo/\td/}}
\input{\repExo/\td.tex} 
 
\renewcommand{\td}{TAB-027}
\graphicspath{{\repStyle/png/}{\repExo/\td/}}
\input{\repExo/\td.tex} 
 

%\end{multicols}
%\newpage
%
%\begin{multicols}{2}
%\section{Chaînes de caractères}
%\renewcommand{\repExo}{../../Informatique/Exercices/04_chaines} 
 
\renewcommand{\td}{STR-000}
\graphicspath{{\repStyle/png/}{\repExo/\td/}}
\input{\repExo/\td.tex} 
 
\renewcommand{\td}{STR-001}
\graphicspath{{\repStyle/png/}{\repExo/\td/}}
\input{\repExo/\td.tex} 
 
\renewcommand{\td}{STR-002}
\graphicspath{{\repStyle/png/}{\repExo/\td/}}
\input{\repExo/\td.tex} 
 
\renewcommand{\td}{STR-003}
\graphicspath{{\repStyle/png/}{\repExo/\td/}}
\input{\repExo/\td.tex} 
 

%\end{multicols}
%\newpage
%
%\begin{multicols}{2}
%\section{Programmation dynamique}
%\renewcommand{\repExo}{../../Informatique/Exercices/05_programmation_dynamique} 
 
\renewcommand{\td}{PDY-000}
\graphicspath{{\repStyle/png/}{\repExo/\td/}}
\input{\repExo/\td.tex} 
 

%\end{multicols}
%\newpage
%
%\begin{multicols}{2}
%\section{Complexité}
%\renewcommand{\repExo}{../../Informatique/Exercices/06_complexite} 
 
\renewcommand{\td}{COM}
\graphicspath{{\repStyle/png/}{\repExo/\td/}}
\input{\repExo/\td.tex} 
 
\renewcommand{\td}{COM-000}
\graphicspath{{\repStyle/png/}{\repExo/\td/}}
\input{\repExo/\td.tex} 
 
\renewcommand{\td}{COM-001}
\graphicspath{{\repStyle/png/}{\repExo/\td/}}
\input{\repExo/\td.tex} 
 
\renewcommand{\td}{COM-002}
\graphicspath{{\repStyle/png/}{\repExo/\td/}}
\input{\repExo/\td.tex} 
 
\renewcommand{\td}{COM}
\graphicspath{{\repStyle/png/}{\repExo/\td/}}
\input{\repExo/\td.tex} 
 
\renewcommand{\td}{COM-003}
\graphicspath{{\repStyle/png/}{\repExo/\td/}}
\input{\repExo/\td.tex} 
 
\renewcommand{\td}{COM-004}
\graphicspath{{\repStyle/png/}{\repExo/\td/}}
\input{\repExo/\td.tex} 
 
\renewcommand{\td}{COM-005}
\graphicspath{{\repStyle/png/}{\repExo/\td/}}
\input{\repExo/\td.tex} 
 
\renewcommand{\td}{COM}
\graphicspath{{\repStyle/png/}{\repExo/\td/}}
\input{\repExo/\td.tex} 
 
\renewcommand{\td}{COM-006}
\graphicspath{{\repStyle/png/}{\repExo/\td/}}
\input{\repExo/\td.tex} 
 
\renewcommand{\td}{COM}
\graphicspath{{\repStyle/png/}{\repExo/\td/}}
\input{\repExo/\td.tex} 
 
\renewcommand{\td}{COM-007}
\graphicspath{{\repStyle/png/}{\repExo/\td/}}
\input{\repExo/\td.tex} 
 
\renewcommand{\td}{COM-008}
\graphicspath{{\repStyle/png/}{\repExo/\td/}}
\input{\repExo/\td.tex} 
 
\renewcommand{\td}{COM-009}
\graphicspath{{\repStyle/png/}{\repExo/\td/}}
\input{\repExo/\td.tex} 
 
\renewcommand{\td}{COM-010}
\graphicspath{{\repStyle/png/}{\repExo/\td/}}
\input{\repExo/\td.tex} 
 
\renewcommand{\td}{COM-011}
\graphicspath{{\repStyle/png/}{\repExo/\td/}}
\input{\repExo/\td.tex} 
 

%\end{multicols}
%\newpage
%
%\begin{multicols}{2}
%\section{Fichiers}
%\renewcommand{\repExo}{../../Informatique/Exercices/10_fichiers} 
 
\renewcommand{\td}{FIC-000}
\graphicspath{{\repStyle/png/}{\repExo/\td/}}
\input{\repExo/\td.tex} 
 
\renewcommand{\td}{FIC-001}
\graphicspath{{\repStyle/png/}{\repExo/\td/}}
\input{\repExo/\td.tex} 

\renewcommand{\td}{FIC-002}
\graphicspath{{\repStyle/png/}{\repExo/\td/}}
\input{\repExo/\td.tex} 
 
\renewcommand{\td}{FIC-003}
\graphicspath{{\repStyle/png/}{\repExo/\td/}}
\input{\repExo/\td.tex} 
 
\renewcommand{\td}{FIC-004}
\graphicspath{{\repStyle/png/}{\repExo/\td/}}
\input{\repExo/\td.tex} 
 
\renewcommand{\td}{FIC-005}
\graphicspath{{\repStyle/png/}{\repExo/\td/}}
\input{\repExo/\td.tex} 
 
\renewcommand{\td}{FIC-006}
\graphicspath{{\repStyle/png/}{\repExo/\td/}}
\input{\repExo/\td.tex} 
 
\renewcommand{\td}{FIC-007}
\graphicspath{{\repStyle/png/}{\repExo/\td/}}
\input{\repExo/\td.tex} 
 
\renewcommand{\td}{FIC-008}
\graphicspath{{\repStyle/png/}{\repExo/\td/}}
\input{\repExo/\td.tex} 
 
\renewcommand{\td}{FIC-009}
\graphicspath{{\repStyle/png/}{\repExo/\td/}}
\input{\repExo/\td.tex} 
 
\renewcommand{\td}{FIC-010}
\graphicspath{{\repStyle/png/}{\repExo/\td/}}
\input{\repExo/\td.tex} 
 
\renewcommand{\td}{FIC-011}
\graphicspath{{\repStyle/png/}{\repExo/\td/}}
\input{\repExo/\td.tex} 
 
\renewcommand{\td}{FIC-012}
\graphicspath{{\repStyle/png/}{\repExo/\td/}}
\input{\repExo/\td.tex} 
 

%\end{multicols}
%\newpage
%
%\begin{multicols}{2}
%\section{Tracer de courbes}
%\renewcommand{\repExo}{../../Informatique/Exercices/11_plot} 
 
\renewcommand{\td}{PLT}
\graphicspath{{\repStyle/png/}{\repExo/\td/}}
\input{\repExo/\td.tex} 
 
\renewcommand{\td}{PLT-000}
\graphicspath{{\repStyle/png/}{\repExo/\td/}}
\input{\repExo/\td.tex} 
 
\renewcommand{\td}{PLT}
\graphicspath{{\repStyle/png/}{\repExo/\td/}}
\input{\repExo/\td.tex} 
 
\renewcommand{\td}{PLT-001}
\graphicspath{{\repStyle/png/}{\repExo/\td/}}
\input{\repExo/\td.tex} 
 
\renewcommand{\td}{PLT-002}
\graphicspath{{\repStyle/png/}{\repExo/\td/}}
\input{\repExo/\td.tex} 
 
\renewcommand{\td}{PLT}
\graphicspath{{\repStyle/png/}{\repExo/\td/}}
\input{\repExo/\td.tex} 
 
\renewcommand{\td}{PLT-003}
\graphicspath{{\repStyle/png/}{\repExo/\td/}}
\input{\repExo/\td.tex} 
 
\renewcommand{\td}{PLT-004}
\graphicspath{{\repStyle/png/}{\repExo/\td/}}
\input{\repExo/\td.tex} 
 
\renewcommand{\td}{PLT-005}
\graphicspath{{\repStyle/png/}{\repExo/\td/}}
\input{\repExo/\td.tex} 
 

%\end{multicols}
%\newpage
%
%\begin{multicols}{2}
%\section{Architecture}
%\renewcommand{\repExo}{../../Informatique/Exercices/20_architecture} 
 
\renewcommand{\td}{ARCHI-000}
\graphicspath{{\repStyle/png/}{\repExo/\td/}}
\input{\repExo/\td.tex} 
 
\renewcommand{\td}{ARCHI-001}
\graphicspath{{\repStyle/png/}{\repExo/\td/}}
\input{\repExo/\td.tex} 
 
\renewcommand{\td}{ARCHI-002}
\graphicspath{{\repStyle/png/}{\repExo/\td/}}
\input{\repExo/\td.tex} 
 
\renewcommand{\td}{ARCHI-004}
\graphicspath{{\repStyle/png/}{\repExo/\td/}}
\input{\repExo/\td.tex} 
 
% \renewcommand{\td}{consignes}
% \graphicspath{{\repStyle/png/}{\repExo/\td/}}
% \input{\repExo/\td.tex} 
 
% \renewcommand{\td}{installation}
% \graphicspath{{\repStyle/png/}{\repExo/\td/}}
% \input{\repExo/\td.tex} 
 

%\end{multicols}
%\newpage
%
%\begin{multicols}{2}
%\section{Représentation des nombres}
%\renewcommand{\repExo}{../../Informatique/Exercices/21_nombres} 
 
\renewcommand{\td}{exo40}
\graphicspath{{\repStyle/png/}{\repExo/\td/}}
\input{\repExo/\td.tex} 
 
\renewcommand{\td}{NBR-000}
\graphicspath{{\repStyle/png/}{\repExo/\td/}}
\input{\repExo/\td.tex} 
 
\renewcommand{\td}{NBR-00}
\graphicspath{{\repStyle/png/}{\repExo/\td/}}
\input{\repExo/\td.tex} 
 
\renewcommand{\td}{NBR-001}
\graphicspath{{\repStyle/png/}{\repExo/\td/}}
\input{\repExo/\td.tex} 
 
\renewcommand{\td}{NBR-00}
\graphicspath{{\repStyle/png/}{\repExo/\td/}}
\input{\repExo/\td.tex} 
 
\renewcommand{\td}{NBR-002}
\graphicspath{{\repStyle/png/}{\repExo/\td/}}
\input{\repExo/\td.tex} 
 
\renewcommand{\td}{NBR-00}
\graphicspath{{\repStyle/png/}{\repExo/\td/}}
\input{\repExo/\td.tex} 
 
\renewcommand{\td}{NBR-003}
\graphicspath{{\repStyle/png/}{\repExo/\td/}}
\input{\repExo/\td.tex} 
 
\renewcommand{\td}{NBR-00}
\graphicspath{{\repStyle/png/}{\repExo/\td/}}
\input{\repExo/\td.tex} 
 
\renewcommand{\td}{NBR-004}
\graphicspath{{\repStyle/png/}{\repExo/\td/}}
\input{\repExo/\td.tex} 
 
\renewcommand{\td}{NBR-00}
\graphicspath{{\repStyle/png/}{\repExo/\td/}}
\input{\repExo/\td.tex} 
 
\renewcommand{\td}{NBR-005}
\graphicspath{{\repStyle/png/}{\repExo/\td/}}
\input{\repExo/\td.tex} 
 

%\end{multicols}
%\newpage
%
%\begin{multicols}{2}
%\section{Intégration numérique}
%\renewcommand{\repExo}{../../Informatique/Exercices/30_integration_numerique} 
 
\renewcommand{\td}{INT-001}
\graphicspath{{\repStyle/png/}{\repExo/\td/}}
\input{\repExo/\td.tex} 
 
\renewcommand{\td}{INT}
\graphicspath{{\repStyle/png/}{\repExo/\td/}}
\input{\repExo/\td.tex} 
 
\renewcommand{\td}{INT-002}
\graphicspath{{\repStyle/png/}{\repExo/\td/}}
\input{\repExo/\td.tex} 
 
\renewcommand{\td}{INT}
\graphicspath{{\repStyle/png/}{\repExo/\td/}}
\input{\repExo/\td.tex} 
 
\renewcommand{\td}{INT-003}
\graphicspath{{\repStyle/png/}{\repExo/\td/}}
\input{\repExo/\td.tex} 
 
\renewcommand{\td}{INT-004}
\graphicspath{{\repStyle/png/}{\repExo/\td/}}
\input{\repExo/\td.tex} 
 
\renewcommand{\td}{periode_pendule}
\graphicspath{{\repStyle/png/}{\repExo/\td/}}
\input{\repExo/\td.tex} 
 

%\end{multicols}
%\newpage
%
%\begin{multicols}{2}
%\section{Équations différentielles}
%\renewcommand{\repExo}{../../Informatique/Exercices/31_equadiffs} 
 
\renewcommand{\td}{EQD-000}
\graphicspath{{\repStyle/png/}{\repExo/\td/}}
\input{\repExo/\td.tex} 
 
\renewcommand{\td}{EQD-001}
\graphicspath{{\repStyle/png/}{\repExo/\td/}}
\input{\repExo/\td.tex} 
 
\renewcommand{\td}{EQD-002}
\graphicspath{{\repStyle/png/}{\repExo/\td/}}
\input{\repExo/\td.tex} 
 
\renewcommand{\td}{EQD-003}
\graphicspath{{\repStyle/png/}{\repExo/\td/}}
\input{\repExo/\td.tex} 
 
\renewcommand{\td}{EQD-004}
\graphicspath{{\repStyle/png/}{\repExo/\td/}}
\input{\repExo/\td.tex} 
 
\renewcommand{\td}{EQD-005}
\graphicspath{{\repStyle/png/}{\repExo/\td/}}
\input{\repExo/\td.tex} 
 
\renewcommand{\td}{EQD-006}
\graphicspath{{\repStyle/png/}{\repExo/\td/}}
\input{\repExo/\td.tex} 
 
\renewcommand{\td}{EQD-007}
\graphicspath{{\repStyle/png/}{\repExo/\td/}}
\input{\repExo/\td.tex} 
 
\renewcommand{\td}{EQD-008}
\graphicspath{{\repStyle/png/}{\repExo/\td/}}
\input{\repExo/\td.tex} 
 
\renewcommand{\td}{maitre_chien}
\graphicspath{{\repStyle/png/}{\repExo/\td/}}
\input{\repExo/\td.tex} 
 

%\end{multicols}
%\newpage
%
%\begin{multicols}{2}
%\section{Équations stationnaires}
%\renewcommand{\repExo}{../../Informatique/Exercices/32_stationnaire} 
 
\renewcommand{\td}{STATIO-001}
\graphicspath{{\repStyle/png/}{\repExo/\td/}}
\input{\repExo/\td.tex} 
 
\renewcommand{\td}{STATIO-002}
\graphicspath{{\repStyle/png/}{\repExo/\td/}}
\input{\repExo/\td.tex} 
 
\renewcommand{\td}{STATIO-003}
\graphicspath{{\repStyle/png/}{\repExo/\td/}}
\input{\repExo/\td.tex} 
 
\renewcommand{\td}{STATIO}
\graphicspath{{\repStyle/png/}{\repExo/\td/}}
\input{\repExo/\td.tex} 
 
\renewcommand{\td}{STATIO-004}
\graphicspath{{\repStyle/png/}{\repExo/\td/}}
\input{\repExo/\td.tex} 
 
\renewcommand{\td}{STATIO}
\graphicspath{{\repStyle/png/}{\repExo/\td/}}
\input{\repExo/\td.tex} 
 
\renewcommand{\td}{STATIO-005}
\graphicspath{{\repStyle/png/}{\repExo/\td/}}
\input{\repExo/\td.tex} 
 

%\end{multicols}
%%\newpage
%
%\begin{multicols}{2}
%\section{Systèmes d'équations}
%\renewcommand{\repExo}{../../Informatique/Exercices/33_systemes} 
 
\renewcommand{\td}{poly}
\graphicspath{{\repStyle/png/}{\repExo/\td/}}
\input{\repExo/\td.tex} 
 
\renewcommand{\td}{polynome}
\graphicspath{{\repStyle/png/}{\repExo/\td/}}
\input{\repExo/\td.tex} 
 
\renewcommand{\td}{pont_de_wheas}
\graphicspath{{\repStyle/png/}{\repExo/\td/}}
\input{\repExo/\td.tex} 
 
\renewcommand{\td}{pont_de_wheastone}
\graphicspath{{\repStyle/png/}{\repExo/\td/}}
\input{\repExo/\td.tex} 
 
\renewcommand{\td}{rob}
\graphicspath{{\repStyle/png/}{\repExo/\td/}}
\input{\repExo/\td.tex} 
 
\renewcommand{\td}{robucar}
\graphicspath{{\repStyle/png/}{\repExo/\td/}}
\input{\repExo/\td.tex} 
 
\renewcommand{\td}{SYS-000}
\graphicspath{{\repStyle/png/}{\repExo/\td/}}
\input{\repExo/\td.tex} 
 
\renewcommand{\td}{SYS}
\graphicspath{{\repStyle/png/}{\repExo/\td/}}
\input{\repExo/\td.tex} 
 
\renewcommand{\td}{SYS-001}
\graphicspath{{\repStyle/png/}{\repExo/\td/}}
\input{\repExo/\td.tex} 
 
\renewcommand{\td}{SYS}
\graphicspath{{\repStyle/png/}{\repExo/\td/}}
\input{\repExo/\td.tex} 
 
\renewcommand{\td}{SYS-002}
\graphicspath{{\repStyle/png/}{\repExo/\td/}}
\input{\repExo/\td.tex} 
 
\renewcommand{\td}{SYS-003}
\graphicspath{{\repStyle/png/}{\repExo/\td/}}
\input{\repExo/\td.tex} 
 
\renewcommand{\td}{SYS-004}
\graphicspath{{\repStyle/png/}{\repExo/\td/}}
\input{\repExo/\td.tex} 
 
\renewcommand{\td}{SYS-005}
\graphicspath{{\repStyle/png/}{\repExo/\td/}}
\input{\repExo/\td.tex} 
 
\renewcommand{\td}{SYS-006}
\graphicspath{{\repStyle/png/}{\repExo/\td/}}
\input{\repExo/\td.tex} 
 
\renewcommand{\td}{SYS-007}
\graphicspath{{\repStyle/png/}{\repExo/\td/}}
\input{\repExo/\td.tex} 
 

%\end{multicols}
%\newpage
%
%\begin{multicols}{2}
%\section{Bases de données}
%\renewcommand{\repExo}{../../Informatique/Exercices/40_sql} 
 
\renewcommand{\td}{enquete}
\graphicspath{{\repStyle/png/}{\repExo/\td/}}
\input{\repExo/\td.tex} 
 
\renewcommand{\td}{hotel}
\graphicspath{{\repStyle/png/}{\repExo/\td/}}
\input{\repExo/\td.tex} 
 
\renewcommand{\td}{projet}
\graphicspath{{\repStyle/png/}{\repExo/\td/}}
\input{\repExo/\td.tex} 
 
\renewcommand{\td}{SQL-000}
\graphicspath{{\repStyle/png/}{\repExo/\td/}}
\input{\repExo/\td.tex} 
 
\renewcommand{\td}{SQL-001}
\graphicspath{{\repStyle/png/}{\repExo/\td/}}
\input{\repExo/\td.tex} 
 
\renewcommand{\td}{SQL-002}
\graphicspath{{\repStyle/png/}{\repExo/\td/}}
\input{\repExo/\td.tex} 
 
\renewcommand{\td}{SQL-003}
\graphicspath{{\repStyle/png/}{\repExo/\td/}}
\input{\repExo/\td.tex} 
 
\renewcommand{\td}{SQL-004}
\graphicspath{{\repStyle/png/}{\repExo/\td/}}
\input{\repExo/\td.tex} 
 
\renewcommand{\td}{SQL-005}
\graphicspath{{\repStyle/png/}{\repExo/\td/}}
\input{\repExo/\td.tex} 
 
\renewcommand{\td}{SQL-006}
\graphicspath{{\repStyle/png/}{\repExo/\td/}}
\input{\repExo/\td.tex} 
 

%\end{multicols}
%\newpage
%
%\begin{multicols}{2}
%\section{Problèmes}
%\renewcommand{\repExo}{../../Informatique/Exercices/50_problemes} 
 
\renewcommand{\td}{puissance_}
\graphicspath{{\repStyle/png/}{\repExo/\td/}}
\input{\repExo/\td.tex} 
 
\renewcommand{\td}{puissance_velo}
\graphicspath{{\repStyle/png/}{\repExo/\td/}}
\input{\repExo/\td.tex} 
 

%\end{multicols}
%\newpage






\end{document}




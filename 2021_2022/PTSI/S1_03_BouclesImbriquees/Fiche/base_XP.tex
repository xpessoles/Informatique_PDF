\usepackage{hyperref}
\hypersetup{
    colorlinks=true,
    linkcolor=blue,
    filecolor=magenta,      
    urlcolor=cyan,
    pdftitle={Overleaf Example},
    pdfpagemode=FullScreen,
    }

%%% Python
\usepackage{listingsutf8}
\lstset{language=Python,
  inputencoding=utf8/latin1,
  breaklines=true,
  basicstyle=\ttfamily\small,
  keywordstyle=\bfseries\color{green!40!black},
  commentstyle=\itshape\color{purple!40!black},
  identifierstyle=\color{blue},
  stringstyle=\color{orange},
  upquote = true,
  columns=fullflexible,
  backgroundcolor=\color{gray!10},frame=leftline,rulecolor=\color{gray}}  
  

\lstset{
     literate=%
         {é}{{\'e}}1    
         {è}{{\`e}}1    
         {ê}{{\^e}}1    
         {à}{{\`a}}1    
         {ô}{{\^o}}1    
         {ù}{{\`u}}1    
}

%%% Exemple
\newenvironment{exemple}[1][\hsize]%
{%
    \def\FrameCommand%
    {%
\rotatebox{90}{\textit{\textsf{Exemple}}} 
        {\color{blue}\vrule width 3pt}%
        \hspace{0pt}%must no space.
        \fboxsep=\FrameSep\colorbox{gris_c}%
    }%
    \MakeFramed{\hsize#1\advance\hsize-\width\FrameRestore}%
}%
{\endMakeFramed}%


%% Questions et exercices
\newcounter{numques}%Création d'un compteur qui s'appelle numques
\setcounter{numques}{0}%initialisation du compteur
\newcommand{\question}[1]{%Création d'une macro ayant un paramètre
\addtocounter{numques}{1}%chaque fois que cette macro est appelée, elle ajoute 1 au compteur numexos
\textbf{Question\, \textcolor{black}{\thenumques}\,}\,\textit{#1}\\{\vspace{0cm}}}